
\chapter{Abelian categories}


This first chapter will serve as an introduction to the basics of Abelian Categories, first introduced by Alexander Grothendieck in his seminal "Tohoku Paper" \cite{Tohoku}.
We will begin with the basic definitions.

\section{Basic definitions}
Let $\C$ be a category.
Let $u:A\longrightarrow B$ and $u':A'\longrightarrow B$ be monomorphisms in $\text{Ar}\: \C$, we say that $u$ majorizes $u'$, written $u \leq u'$, if there exists a morphism $v:A \longrightarrow A'$ such that $u=u'\circ v$. This relation defines a preorder on the class of arrows into $B$, from which we can extract a equivalence relation



\begin{definition}[Subobject]
	
\end{definition}



\section{Additive categories}


\begin{definition}[Additive category]
    A \textbf{additive category} is a (locally small) category $\mathcal{C}$ such that:
    \begin{itemize}
        \item for every couple of objects $A,B \in \ob \: \C$ the set of arrows $\Hom_\C(A,B)$ is an abelian group such that composition distributes over its group operation;
        \item $\C$ has binary products.
        \item $\C$ has a zero object, that is an object that is both initial and terminal.
    \end{itemize} 
    We will refer to the identity element of $\Hom_\C(A,B)$ as the $0$-map.
\end{definition}

We will now let $\C$ be an additive category.

\begin{remark}
	From the distributivity of composition we deduce that composition must be bilinear; therefore for every triplet of objects $A,B,C$ we get a morphism of abelian groups
	\begin{align}
		\Hom(A,B) \otimes \Hom(B,C) & \longrightarrow \Hom(A, C) \\
		f \otimes g & \mapsto gf.
	\end{align}
	An immediate consequence of this fact is that the composition of $0$-maps is the $0$-map.
\end{remark}


Let us now consider two objects $A,B$ in $\C$, we must have a (unique) terminal map $A \longrightarrow 0$ and a (unique) initial map $0 \longrightarrow B$; by uniqueness they must both be $0$-maps, so their composition is the zero map. We have shown that the $0$-map is the unique map that factors through the zero object.



We can easily check that every binary product $A \times B$ is also a binary coproduct, where the canonical maps $i_A$, $i_B$ are given by $(id_A, 0)$ and $(0, id_B)$.

\[\begin{tikzcd}
	& A \\
	& {A\times B} \\
	A && B
	\arrow["{i_A}", dashed, from=1-2, to=2-2]
	\arrow["{id_A}"', from=1-2, to=3-1]
	\arrow["0", from=1-2, to=3-3]
	\arrow["{\pi_A}", from=2-2, to=3-1]
	\arrow["{\pi_B}"', from=2-2, to=3-3]
\end{tikzcd}\]

We will denote the object that is both the product and the coproduct with the direct sum symbol, as an example we have $A\times B \cong A \amalg B \cong A \oplus B$.

\begin{definition}[Kernel and cokernel]
	Let $f : A\longrightarrow B$ be an arrow of $\C$, the kernel of $f$ is the equalizer 
	\[\begin{tikzcd}
	Ker(f) & A & B, & {}
	\arrow["0"', shift right=1, from=1-2, to=1-3]
	\arrow["f", shift left=1, from=1-2, to=1-3]
	\arrow[from=1-1, to=1-2]
	\end{tikzcd}\]
	while the cokernel is the coequalizer of the diagram
	\[\begin{tikzcd}
	 A & B & Coker(f). & {}
	\arrow["0"', shift right=1, from=1-1, to=1-2]
	\arrow["f", shift left=1, from=1-1, to=1-2]
	\arrow[from=1-2, to=1-3]
	\end{tikzcd}\]
	We will often refer to both the object and the arrow of the coequalizer as the kernel. We also define the image of $f$, denoted $Im f$, as $Ker(Coker(f))$ and the coimage of $f$, denoted $Coim(f)$, as $Coker(Ker(f))$.

\end{definition}

Applying the relevant universal properties it is easy to see that there exist a canonical morphism $Coim(f) \longrightarrow Im(f)$ such that the following diagram commutes
\[\begin{tikzcd}
	{Ker(f)} & A && B & {Coker(f)} \\
	& {Coim(f)} && {Im(f).}
	\arrow[from=1-1, to=1-2]
	\arrow["f", from=1-2, to=1-4]
	\arrow[from=1-2, to=2-2]
	\arrow[from=1-4, to=1-5]
	\arrow[from=2-2, to=2-4]
	\arrow[from=2-4, to=1-4]
\end{tikzcd}\]

\begin{definition}[Additive functor]
	An \textbf{additive functor} $F: \C \longrightarrow \mathcal{D}$ is a functor between additive categories such that for every pair of objects $A,B \in \ob \: \C$ the induced map
	\[F_{A,B}:\Hom_\C(A,B) \longrightarrow \Hom_{\mathcal{D}}(FA,FB)\] 
	is a group homomorphism.
\end{definition}





\section{Abelian Categories}

\begin{definition}[Abelian category]
	An abelian category is an additive category $\C$ that satisfies the following axioms:
	\begin{itemize}
  		\item[(AB1)] any morphism admits a kernel and a cokernel;
    	\item[(AB2)] for every morphism $f:A\longrightarrow B$ the canonical map $Coim(f) \longrightarrow Im(f)$ is an isomorphism.
\end{itemize}
\end{definition}

