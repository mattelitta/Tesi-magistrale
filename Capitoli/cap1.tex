
\chapter{Abelian categories}


This first chapter will serve as an introduction to the basics of Abelian Categories, first introduced by Alexander Grothendieck in his seminal "Tohoku Paper" \cite{Tohoku}.
We will begin with the basic definitions.

\section{Basic definitions}

\begin{definition}
    A \textbf{additive category} is a (locally small) category $\mathcal{C}$ such that:
    \begin{itemize}
        \item for every couple of objects $A,B \in \ob \: \C$ the set of arrows $\Hom_\C(A,B)$ is an abelian group;
        \item $\C$ has binary products.
        \item $\C$ has a zero object, that is an object that is both initial and terminal.
    \end{itemize} 
    We will refer to the identity element of $\Hom_\C(A,B)$ as the $0$-map.
\end{definition}


We can easily check that every binary product $A \times B$ is also a binary coproduct, where the canonical maps $i_A$, $i_B$ are given by $(id_A, 0)$ and $(0, id_B)$.

\[\begin{tikzcd}
	& A \\
	& {A\times B} \\
	A && B
	\arrow["{i_A}", dashed, from=1-2, to=2-2]
	\arrow["{id_A}"', from=1-2, to=3-1]
	\arrow["0", from=1-2, to=3-3]
	\arrow["{\pi_A}", from=2-2, to=3-1]
	\arrow["{\pi_B}"', from=2-2, to=3-3]
\end{tikzcd}\]

We will denote the object that is both the product and the coproduct with the direct sum symbol, as an example we have $A\times B \cong A \amalg B \cong A \oplus B$.
