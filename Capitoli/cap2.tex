
\chapter{Abelian categories}


This first chapter will serve as an introduction to the basics of Abelian Categories, first introduced by Alexander Grothendieck in his seminal "Tohoku Paper" \cite{Tohoku}; for a nice introduction see \cite{murfet2006abelian}.

\section{Basic definitions}
Let $\C$ be a category.
Let $u\colon A\longrightarrow B$ and $u'\colon A'\longrightarrow B$ be monomorphisms in $\text{Ar}\: \C$. We say that $u$ majorizes $u'$, written $u \leq u'$, if there exists a morphism $v:A \longrightarrow A'$ such that $u=u'\circ v$.\\
 This relation defines a preorder on the class of arrows into $B$, from which we can extract an equivalence relation: $u \cong v$ if and only if $u \leq v$ and $v \leq u$. 



\begin{definition}[Subobject]
	Let $B \in \ob \: \C$, we define a \textbf{subobject} of $B$ to be an equivalence class of monomorphisms into $B$.
\end{definition}

\begin{remark}
	While taking a quotient, we could run into size issues when $\Hom_\C(A,B)$ is a proper class; we will generally avoid discussing set-theoretic issues.
\end{remark}

\begin{remark}
	Let $A$ and $B$ be representatives of the same subobject of $X$, then we have a diagram
	\[\begin{tikzcd}
		& X \\
		A & {B.}
		\arrow["a", hook, from=2-1, to=1-2]
		\arrow["u", shift left, from=2-1, to=2-2]
		\arrow["b"', hook', from=2-2, to=1-2]
		\arrow["v", shift left, from=2-2, to=2-1]
	\end{tikzcd}\]
	By applying the definition of monomorphism, it is easy to see that $vu = \id_A$ and $uv = \id_B$ (also: $u$ and $v$ are unique), therefore $A$ and $B$ are isomorphic. This implies that being the same subobject is equivalent to being isomorphic in the slice category $\C / X$.
\end{remark}

We can now discuss the operation we can take between subobjects; every construction that we will describe is not dependent on the representative of the equivalence class of monomorphisms.

\begin{definition}[Intersection and union]
	Let $A \longrightarrow X$ and $B\longrightarrow X$ be subobjects, we define the \textbf{intersection} $A \cap B$ as the following pullback (when it exists):

	\[\begin{tikzcd}
		& {A\cap B} \\
		A && B \\
		& X.
		\arrow[dashed, from=1-2, to=2-1]
		\arrow[dashed, from=1-2, to=2-3]
		\arrow[dashed, from=1-2, to=3-2]
		\arrow[from=2-1, to=3-2]
		\arrow[from=2-3, to=3-2]
	\end{tikzcd}\]
	We can also define the \textbf{sum} $A+B$ of two subobjects as the following pushout (when it exists):
	\[\begin{tikzcd}
		& {A\cap B} \\
		A && B \\
		& {A+B} \\
		& {X.}
		\arrow[from=1-2, to=2-1]
		\arrow[from=1-2, to=2-3]
		\arrow[dashed, from=2-1, to=3-2]
		\arrow[from=2-1, to=4-2]
		\arrow[dashed, from=2-3, to=3-2]
		\arrow[from=2-3, to=4-2]
		\arrow[dashed, from=3-2, to=4-2]
	\end{tikzcd}\]

\end{definition}


\begin{remark}
	The preorder relation on monomorphisms induces a partial order on the subobjects of a specific object; the intersection is the meet of this poset, while the sum is the join.
\end{remark}

We can dualize all the previous constructions, to obtain the following definition.

\begin{definition}[Quotient]
	Let $B \in \ob \: \C$, we define a \textbf{quotient} of $B$ to be an equivalence class of epimorphisms from $B$.
\end{definition}

All the dual results obviously hold.

\section{Additive categories}

We start this section introducing additive categories and functors between them.

\begin{definition}[Additive category]
    A \textbf{additive category} is a (locally small) category $\mathcal{C}$ such that:
    \begin{itemize}
        \item for every couple of objects $A,B \in \ob \: \C$ the set of arrows $\Hom_\C(A,B)$ is an abelian group such that composition distributes over its group operation;
        \item $\C$ has binary products.
        \item $\C$ has a zero object, that is an object that is both initial and terminal.
    \end{itemize} 
    We will refer to the identity element of $\Hom_\C(A,B)$ as the $0$-morphism.
\end{definition}


\begin{definition}[Additive functor]
	An \textbf{additive functor} $F: \C \longrightarrow \mathcal{D}$ is a functor between additive categories such that for every pair of objects $A,B \in \ob \: \C$ the induced morphism
	\[F_{A,B}:\Hom_\C(A,B) \longrightarrow \Hom_{\mathcal{D}}(FA,FB)\] 
	is a group homomorphism.
\end{definition}


We will now let $\C$ be an additive category.

\begin{remark}
	From the distributivity of composition we deduce that composition must be bilinear; therefore for every triplet of objects $A,B,C$ we get a morphism of abelian groups
	\begin{align}
		\Hom_\C(A,B) \otimes \Hom_\C(B,C) & \longrightarrow \Hom_\C(A, C) \\
		f \otimes g & \mapsto gf.
	\end{align}
	An immediate consequence of this fact is that the composition of $0$-morphisms is the $0$-morphism.
\end{remark}


\begin{proposition}
	\label{prop:mono-equiv}
	Let $f$ be a morphism in $\C$, the following are equivalents:
	\begin{itemize}
		\item $f$ is monic;
		\item $fg = 0$ implies $g=0$ for every composable $g$.
	\end{itemize}
\end{proposition}


\begin{proof}
	Recall that the monomorphisms between two objects $A,B$ are exactly the morphisms $f$ such that for every morphism $h:C \longrightarrow B$ we have at most one morphism $g:C \longrightarrow B$ such that the following diagram commutes:

	\[\begin{tikzcd}
		A & B \\
		C.
		\arrow["f", hook, from=1-1, to=1-2]
		\arrow["g", from=2-1, to=1-1]
		\arrow["h"', from=2-1, to=1-2]
	\end{tikzcd}\]
	Let $g_1,g_2$ be two such morphisms, then the commutativity of the diagram above is equivalent to requiring that $fg_1 = h = fg_2$, which implies $f(g_1 - g_2) = 0$. Given that $h$ can vary between al morphisms that factor through $f$, we can set $g = g_1 - g_2$ and get the thesis.

\end{proof}

The dual of the above proposition is also true.

\begin{proposition}
	Let $f$ be a morphism in $\C$, the following are equivalents:
	\begin{itemize}
		\item $f$ is epic;
		\item $gf = 0$ implies $g=0$ for every composable $g$.
	\end{itemize}
\end{proposition}


Suppose that $A,B$ are two objects in $\C$, we must have a (unique) terminal morphism $A \longrightarrow 0$ and a (unique) initial morphism $0 \longrightarrow B$; by uniqueness they must both be $0$-morphisms, so their composition is the zero morphism. We have shown that the $0$-morphism is the unique morphism that factors through the zero object.



We can easily check that every binary product $A \times B$ is also a binary coproduct, where the canonical morphisms $\inn_1$, $\inn_2$ are given by $(\id_A, 0)$ and $(0, \id_B)$.


\[\begin{tikzcd}
	& A \\
	& {A\times B} \\
	A && B
	\arrow["{\inn_1}", dashed, from=1-2, to=2-2]
	\arrow["{\id_A}"', from=1-2, to=3-1]
	\arrow["0", from=1-2, to=3-3]
	\arrow["{\pi_A}", from=2-2, to=3-1]
	\arrow["{\pi_B}"', from=2-2, to=3-3]
\end{tikzcd}\]

We will denote the object that is both the product and the coproduct with the direct sum symbol: we have $A\times B \cong A \amalg B \cong A \oplus B$, along all the standard relationships between the canonical morphisms.

\[\begin{tikzcd}
	A && B && A && B \\
	& {A \oplus B} &&&& {A \oplus B} \\
	A && B && B && A
	\arrow["{\inn_1}"', from=1-1, to=2-2]
	\arrow["{\id_A}", from=1-1, to=3-1]
	\arrow["{\inn_2}", from=1-3, to=2-2]
	\arrow["{\id_B}", from=1-3, to=3-3]
	\arrow["{\inn_1}"', from=1-5, to=2-6]
	\arrow["0", from=1-5, to=3-5]
	\arrow["{\inn_2}", from=1-7, to=2-6]
	\arrow["0", from=1-7, to=3-7]
	\arrow["{\pi_1}", from=2-2, to=3-1]
	\arrow["{\pi_2}"', from=2-2, to=3-3]
	\arrow["{\pi_2}", from=2-6, to=3-5]
	\arrow["{\pi_1}"', from=2-6, to=3-7]
\end{tikzcd}\]

We also denote the morphisms produced from $f,g$ (resp. $f',g'$) by the universal property of the (co)product as $(f,g)$ (resp. $f+g$), taking care to note that $f+g$ has not (yet) any relation with the sum in $\Hom(A\oplus B, B)$. 

\[\begin{tikzcd}
	& X &&& A & {A\oplus B} & B \\
	A & {A \oplus B} & B &&& Y
	\arrow["f"', from=1-2, to=2-1]
	\arrow["{(f,g)}", dashed, from=1-2, to=2-2]
	\arrow["g", from=1-2, to=2-3]
	\arrow["{\inn_1}", from=1-5, to=1-6]
	\arrow["{f'}"', from=1-5, to=2-6]
	\arrow["{f+g}", dashed, from=1-6, to=2-6]
	\arrow["{\inn_2}"', from=1-7, to=1-6]
	\arrow["{g'}", from=1-7, to=2-6]
	\arrow["{\pi_1}", from=2-2, to=2-1]
	\arrow["{\pi_2}"', from=2-2, to=2-3]
\end{tikzcd}\]

We will also denote the diagonal morphism $(\id_A, \id_A)$ as $\Delta_A \colon A \longrightarrow A \otimes A$ and the codiagonal morphism $id_B + id_B$ as $\nabla_B \colon B\oplus B \longrightarrow B$.\\
With an analogous construction we get direct sum of morphisms $f \oplus g$, with all the relevant properties.


\[\begin{tikzcd}
	& A & B \\
	{A\oplus A'} &&& {B\oplus B'} \\
	& {A'} & {B'}
	\arrow["f", from=1-2, to=1-3]
	\arrow[from=1-3, to=2-4]
	\arrow["{\pi_1}", from=2-1, to=1-2]
	\arrow["{f \oplus g}", from=2-1, to=2-4]
	\arrow["{\pi_2}"', from=2-1, to=3-2]
	\arrow["g"', from=3-2, to=3-3]
	\arrow[from=3-3, to=2-4]
\end{tikzcd}\]


\[\begin{tikzcd}
	A && {} & B \\
	& {A\oplus A'} & {B\oplus B'} \\
	{A'} &&& {B'}
	\arrow["f", from=1-1, to=1-4]
	\arrow["{\inn_1}"', from=1-1, to=2-2]
	\arrow["{f\oplus g}", from=2-2, to=2-3]
	\arrow["{\pi_1}"', from=2-3, to=1-4]
	\arrow["{\pi_2}", from=2-3, to=3-4]
	\arrow["{\inn_2}", from=3-1, to=2-2]
	\arrow["g", from=3-1, to=3-4]
\end{tikzcd}\]




\begin{remark}
	Let us consider the morphism $\pi_1+\pi_2\colon B\oplus B \longrightarrow B$ given by the additive structure on $\Hom(B\oplus B, B)$; we have that $(\pi_1 +\pi_2)i_1 = \pi_1 \inn_1 + \pi_2 \inn_1 = \id_B + 0 =\id_B $ and that $(\pi_1 +\pi_2)i_2 =\id_B $, therefore we can write $\nabla_B = \pi_1+\pi_2$. A consequence of this result is that, given two morphisms $f,g \colon A \longrightarrow B$, we have
	\[f+g = \nabla_A (f \oplus g) \Delta_B.\]
\end{remark}

\begin{definition}[Kernel and cokernel]
	Let $f : A\longrightarrow B$ be an arrow of $\C$, the \textbf{kernel} of $f$ is the equalizer 
	\[\begin{tikzcd}
	\ker(f) & A & B, & {}
	\arrow["0"', shift right=1, from=1-2, to=1-3]
	\arrow["f", shift left=1, from=1-2, to=1-3]
	\arrow[from=1-1, to=1-2]
	\end{tikzcd}\]
	while the \textbf{cokernel} is the coequalizer of the diagram
	\[\begin{tikzcd}
	 A & B & \coker(f). & {}
	\arrow["0"', shift right=1, from=1-1, to=1-2]
	\arrow["f", shift left=1, from=1-1, to=1-2]
	\arrow[from=1-2, to=1-3]
	\end{tikzcd}\]
	We will often refer to both the object and the arrow of the coequalizer as the kernel. We also define the image of $f$, denoted $Im f$, as $\ker(\coker(f))$ and the coimage of $f$, denoted $\coim(f)$, as $\coker(\ker(f))$.

\end{definition}

\begin{example}
	let $0\colon A \longrightarrow B$ be the $0$ morphism, then $\ker(0) = A$ and $\coker(0) = B$ (with the identity morphism).
\end{example}


\begin{remark}
	Due to the fact that every equalizer is monic (and every coequalizer is epic), it is immediate to see that the kernel of a morphism is a subobject, while the cokernel is a quotient. This also implies that the image is a subobject while the coimage is a quotient.
\end{remark}

Applying the relevant universal properties it is easy to see that there exist a canonical morphism $\coim(f) \longrightarrow \im(f)$ such 
\[\begin{tikzcd}
	{\ker(f)} & A && B & {\coker(f)} \\
	& {Coim(f)} && {Im(f)}
	\arrow[from=1-1, to=1-2]
	\arrow["f", from=1-2, to=1-4]
	\arrow[from=1-2, to=2-2]
	\arrow[from=1-4, to=1-5]
	\arrow[from=2-2, to=2-4]
	\arrow[from=2-4, to=1-4]
\end{tikzcd}\]
commutes.





\section{Abelian Categories}

\begin{definition}[Abelian category]
	An \textbf{abelian category} is an additive category $\C$ that satisfies the following axioms:
	\begin{itemize}
  		\item[(AB1)] any morphism admits a kernel and a cokernel;
    	\item[(AB2)] for every morphism $f:A\longrightarrow B$, the canonical morphism $\coim(f) \longrightarrow \im(f)$ is an isomorphism.
	\end{itemize}
\end{definition}

The existence of the kernel and the cokernel is remarkably powerful: the equalizer of $f$ and $g$ is exactly the kernel of $f-g$, while their coequalizer is its cokernel; therefore in any abelian category equalizers and coequalizers must exist. It is a well known result that it is possible to construct a finite limit using equalizers and finite products, thus every abelian category has finite limits and (dually) finite colimits. One consequence of this result is the fact that we can always take the intersection and the sum of subobjects, therefore the subobjects of a given object form a lattice.


\begin{remark}
	Let $f$ be a monomorphism, because of proposition \ref{prop:mono-equiv} every morphisms that equalizes $f$ and the $0$-morphism must be the $0$-morphism itself; given that we also know that the $0$-morphism factors through the $0$-object, we conclude that the kernel of $f$ must be $0$, and that this is a sufficient condition for $f$ to be monic.\\
	The same can be said for the cokernel of an epimorphism. 
\end{remark}


From the previous remark follows that the coimage of the monomorphism $f: A \longrightarrow B $ is exactly $A$ (or more precisely the identity morphism $A \longrightarrow A$), therefore by applying (AB2) we deduce that every monomorphism is its image and therefore that it is a kernel (as a subobject).

\[\begin{tikzcd}
	0 & A & B & {\coker(f)} \\
	& A & {\im(f)}
	\arrow[from=1-1, to=1-2]
	\arrow[from=1-2, to=1-3]
	\arrow["{\id_A}"', from=1-2, to=2-2]
	\arrow[from=1-3, to=1-4]
	\arrow["\cong", from=2-2, to=2-3]
	\arrow[from=2-3, to=1-3]
\end{tikzcd}\]
Dually every epimorphism is its coimage and therefore it is a cokernel. \\
We continue by introducing one of the main object used in the study of abelian categories: exact sequences.


\begin{lemma}
	Let us consider the following commutative diagram:
	\[\begin{tikzcd}
		W & Y & {} \\
		X & {Z.}
		\arrow["g", from=1-1, to=1-2]
		\arrow["f"', from=1-1, to=2-1]
		\arrow["k", from=1-2, to=2-2]
		\arrow["h"', from=2-1, to=2-2]
	\end{tikzcd}\]
	The following is true:
	\begin{enumerate}
		\item The diagram is a pullback diagram if and only if
		\[\begin{tikzcd}
			0 & W & {X\oplus Y} & Z
			\arrow[from=1-1, to=1-2]
			\arrow["{(f,g)}", from=1-2, to=1-3]
			\arrow["{h-k}", from=1-3, to=1-4]
		\end{tikzcd}\]
		is exact.
		\item The diagram is a pushout diagram if and only if
		\[\begin{tikzcd}
			W & {X\oplus Y} & Z & 0
			\arrow["{(f,-g)}", from=1-1, to=1-2]
			\arrow["{h+k}", from=1-2, to=1-3]
			\arrow[from=1-3, to=1-4]
		\end{tikzcd}\]
		is exact.
	\end{enumerate}
\end{lemma}

\begin{proof}
	\textbf{DA FARE}
\end{proof}

\begin{proposition}
	\label{prop:epi-pull}
	Epimorphism are stable under pullback (and correspond to pushouts), while monomorphisms are stable under pushout (and correspond to pullbacks).
\end{proposition}
\begin{proof}
	\textbf{DA FARE}
\end{proof}



\begin{definition}[Exact sequence]
	Let $f\colon A\longrightarrow B$ and $g\colon B \longrightarrow C$ be morphisms in $\C$, we say that the sequence 
	\[\begin{tikzcd}
		A & B & C
		\arrow["f", from=1-1, to=1-2]
		\arrow["g", from=1-2, to=1-3]
	\end{tikzcd}\]
	is \textbf{exact} if $Im(f)$ and $\ker(g)$ are the same subobject of $B$. More generally if we have a succession of morphism 
	\[\begin{tikzcd}
		\cdots & {A_{k-1}} & {A_k} & {A_{k+1}} & \cdots
		\arrow[from=1-1, to=1-2]
		\arrow["{f_{k-1}}", from=1-2, to=1-3]
		\arrow["{f_k}", from=1-3, to=1-4]
		\arrow[from=1-4, to=1-5]
	\end{tikzcd}\]
	we say that it is exact at $A_k$ if $A_{k-1}\longrightarrow A_k \longrightarrow A_{k+1}$ is exact; we say that it is exact if it is exact at $A_k$ for every $k$. A \textbf{short exact sequence} is a sequence of the form
	\[\begin{tikzcd}
		0 & A & B & C & 0
		\arrow[from=1-1, to=1-2]
		\arrow[from=1-2, to=1-3]
		\arrow[from=1-3, to=1-4]
		\arrow[from=1-4, to=1-5]
	\end{tikzcd}\]
	that is exact at $A$, $B$ and $C$; we will often shorten the name short exact sequence as SES.
\end{definition}

Let us now consider a SES
\[\begin{tikzcd}
	0 & A & B & C & {0;}
	\arrow[from=1-1, to=1-2]
	\arrow["i", from=1-2, to=1-3]
	\arrow["p", from=1-3, to=1-4]
	\arrow[from=1-4, to=1-5]
\end{tikzcd}\]
exactness in $A$ implies that $\ker(i) = \im(0) = 0$, therefore $i$ is monic; similarly $p$ is epic. \\
If we take a monomorphism $i\colon A \longrightarrow B$, we can obtain a SES 
\[\begin{tikzcd}
	0 & A & B & {\coker(i)} & {0.}
	\arrow[from=1-1, to=1-2]
	\arrow["i", from=1-2, to=1-3]
	\arrow[from=1-3, to=1-4]
	\arrow[from=1-4, to=1-5]
\end{tikzcd}\]
Similarly for an epimorphism $p$ we get
\[\begin{tikzcd}
	0 & {\ker(p)} & B & C & {0.}
	\arrow[from=1-1, to=1-2]
	\arrow[from=1-2, to=1-3]
	\arrow["p", from=1-3, to=1-4]
	\arrow[from=1-4, to=1-5]
\end{tikzcd}\]


\begin{remark}
	
\end{remark}

\section{Extensions}
In this section we will essentially follow \cite{Weibel_1994} and \cite{maclane2012homology}.
Let $\C$ be an abelian category, let $A,B$ be objects of $\C$.

\begin{definition}[Extension]
	An \textbf{extension} $\xi$ of $A$ and $B$ is a short exact sequence of the form
	\[\begin{tikzcd}
		{\xi\colon} & 0 & B & X & A & {0.}
		\arrow[from=1-2, to=1-3]
		\arrow[from=1-3, to=1-4]
		\arrow[from=1-4, to=1-5]
		\arrow[from=1-5, to=1-6]
	\end{tikzcd}\]
	More generally, for $n\geq 1$, an $n$-\textbf{extension} $\xi$ of $A$ and $B$ is an exact sequence 
	\[\begin{tikzcd}[sep=small]
		{\xi \colon} & 0 & B & {X_{n-1}} & {X_{n-2}} & \cdots & {X_0} & A & 0
		\arrow[from=1-2, to=1-3]
		\arrow[from=1-3, to=1-4]
		\arrow[from=1-4, to=1-5]
		\arrow[from=1-5, to=1-6]
		\arrow[from=1-6, to=1-7]
		\arrow[from=1-7, to=1-8]
		\arrow[from=1-8, to=1-9]
	\end{tikzcd}\]
\end{definition}

We can now define an equivalence relation on the class of $n$-extensions as follows: we say that the extension $\xi$ and $\xi'$ are equivalent if there is a commutative diagram

\[\begin{tikzcd}[sep=small]
	{\xi \colon} & 0 & B & {X_{n-1}} & {X_{n-2}} & \cdots & {X_0} & A & 0 \\
	& 0 & B & {X''_{n-1}} & {X''_{n-2}} & \cdots & {X''_0} & A & {0} \\
	{\xi'\colon} & 0 & A & {X_{n-1}'} & {X_{n-2}'} & \cdots & {X'_0} & B & {0.}
	\arrow[from=1-2, to=1-3]
	\arrow[from=1-3, to=1-4]
	\arrow[from=1-4, to=1-5]
	\arrow[from=1-5, to=1-6]
	\arrow[from=1-6, to=1-7]
	\arrow[from=1-7, to=1-8]
	\arrow[from=1-8, to=1-9]
	\arrow[from=2-2, to=2-3]
	\arrow["{\id_B}", from=2-3, to=1-3]
	\arrow[from=2-3, to=2-4]
	\arrow["{\id_B}"', from=2-3, to=3-3]
	\arrow[from=2-4, to=1-4]
	\arrow[from=2-4, to=2-5]
	\arrow[from=2-4, to=3-4]
	\arrow[from=2-5, to=1-5]
	\arrow[from=2-5, to=2-6]
	\arrow[from=2-5, to=3-5]
	\arrow[from=2-6, to=2-7]
	\arrow[from=2-7, to=1-7]
	\arrow[from=2-7, to=2-8]
	\arrow[from=2-7, to=3-7]
	\arrow["{\id_A}", from=2-8, to=1-8]
	\arrow[from=2-8, to=2-9]
	\arrow["{\id_A}"', from=2-8, to=3-8]
	\arrow[from=3-2, to=3-3]
	\arrow[from=3-3, to=3-4]
	\arrow[from=3-4, to=3-5]
	\arrow[from=3-5, to=3-6]
	\arrow[from=3-6, to=3-7]
	\arrow[from=3-7, to=3-8]
	\arrow[from=3-8, to=3-9]
\end{tikzcd}\]

\begin{remark}
	If $n=1$, the relation is much simpler: in the following diagram both middle morphisms must necessarily be isomorphism, therefore we need only to look for a morphism $X \to X'$ that makes sure that the following diagram commutes.
	\[\begin{tikzcd}[sep=small]
		{\xi \colon} & 0 & B & X & A & 0 \\
		{\xi'\colon} & 0 & B & {X'} & A & 0
		\arrow[from=1-2, to=1-3]
		\arrow[from=1-3, to=1-4]
		\arrow[from=1-4, to=1-5]
		\arrow[from=1-5, to=1-6]
		\arrow[from=2-2, to=2-3]
		\arrow["{\id_B}", from=1-3, to=2-3]
		\arrow[from=2-3, to=2-4]
		\arrow[from=1-4, to=2-4]
		\arrow[from=2-4, to=2-5]
		\arrow["{\id_A}", from=1-5, to=2-5]
		\arrow[from=2-5, to=2-6]
	\end{tikzcd}\]

\end{remark}

\begin{definition}[Split extensions]
	We say that an extension of $A$ and $B$ is \textbf{split} if it is equivalent to
	\[\begin{tikzcd}
	0 & B & {B\oplus A} & A & 0.
	\arrow[from=1-1, to=1-2]
	\arrow["{i_A}", from=1-2, to=1-3]
	\arrow["{\pi_B}", from=1-3, to=1-4]
	\arrow[from=1-4, to=1-5]
\end{tikzcd}\]
\end{definition}


It is easy to see that an extension $B\longrightarrow X \longrightarrow A$ is split if and only if there exists an isomorphism $f \colon X \longrightarrow B \oplus A$ such that
\[\begin{tikzcd}
	0 & B & X & A & 0 \\
	&& {B \oplus A}
	\arrow[from=1-1, to=1-2]
	\arrow["i", from=1-2, to=1-3]
	\arrow["{\inn_1}"', from=1-2, to=2-3]
	\arrow[from=1-3, to=1-4]
	\arrow["\cong", "f"', from=1-3, to=2-3]
	\arrow["p", from=1-4, to=1-5]
	\arrow["{\pi_2}"', from=2-3, to=1-4]
\end{tikzcd}\]
commutes. From this formulation follows that an extension is split if and only if there exists a retraction of $i$ (i.e. a morphism $q\colon X \longrightarrow B$ such that $qi = \id_B$) or a section of $p$ (i.e. a morphism $j\colon A \longrightarrow X$ such that $pj = \id_A$).

\begin{definition}[Ext]
	We define $\Ext^n_\C(A,B)$ to be quotient of the family of all $n$-extensions by the relation of equivalence. We will also define $\Ext^0_\C(A,B)$ as $\Hom_\C(A,B)$. 
\end{definition}
\begin{remark}
	$\Ext^n_\C(A,B)$ is not necessarily small (i.e. a set), see \cite[\href{https://stacks.math.columbia.edu/tag/07JS}{Section 07JS}]{stacks-project}, but it will be in the relevant cases (Grothendieck categories). From now on, in this section, we will assume it is a set. 
\end{remark}
\begin{remark}
	It is clear that $\Ext^0_\C(A,B)$ is an abelian group, given that $\C$ is an additive category. Our objective will be to show that this is always the case.
\end{remark}

We will show that $\Ext$ is actually a functor from $\C\op \times \C$ to $\cat{Set}$, in the case $n=1$ (for the general case see \cite{maclane2012homology}). 
If we are given an extension 
\[0 \longrightarrow B \longrightarrow X \longrightarrow A \longrightarrow 0 \]
and $h\colon B\longrightarrow B'$ we can take the pushout ${B'\times_B X}$; then, applying the relevant universal property to the morphisms $0\colon B' \longrightarrow A$ and $g \colon X \longrightarrow A$, we get a commuting diagrams with exact lines:


\[\begin{tikzcd}
	0 & B & X & A & 0 \\
	0 & {B'} & {B'\amalg_B X} & A & {0.}
	\arrow[from=1-1, to=1-2]
	\arrow["f", from=1-2, to=1-3]
	\arrow["h", from=1-2, to=2-2]
	\arrow["g", from=1-3, to=1-4]
	\arrow[from=1-3, to=2-3]
	\arrow[from=1-4, to=1-5]
	\arrow["{\id_A}", from=1-4, to=2-4]
	\arrow[from=2-1, to=2-2]
	\arrow["{f'}", from=2-2, to=2-3]
	\arrow["{g'}", from=2-3, to=2-4]
	\arrow[from=2-4, to=2-5]
\end{tikzcd}\]
Therefore, by passing to the quotient, we get a (natural) map \[h_{*}\colon \Ext^1(A,B) \longrightarrow \Ext(A,B').\] 
Similarly, given a morphism $k:A'\longrightarrow A$, we get a (natural) map 
\[k_* \colon \Ext^1_\C(A, B) \longrightarrow \Ext^1_\C(A',B).\] 

We will now introduce an operation, called the \textit{Baer sum}, on $\Ext^n_\C(A,B)$: given two elements in $\Ext^n_\C(A,B)$ with representatives

\[\begin{tikzcd}[sep=small]
	{\xi \colon} & 0 & B & {X_{n-1}} & {X_{n-2}} & \cdots & {X_0} & A & 0 \\
	{\xi' \colon} & 0 & B & {X'_{n-1}} & {X'_{n-2}} & \cdots & {X'_0} & A & 0
	\arrow[from=1-2, to=1-3]
	\arrow[from=1-3, to=1-4]
	\arrow[from=1-4, to=1-5]
	\arrow[from=1-5, to=1-6]
	\arrow[from=1-6, to=1-7]
	\arrow[from=1-7, to=1-8]
	\arrow[from=1-8, to=1-9]
	\arrow[from=2-2, to=2-3]
	\arrow[from=2-3, to=2-4]
	\arrow[from=2-4, to=2-5]
	\arrow[from=2-5, to=2-6]
	\arrow[from=2-6, to=2-7]
	\arrow[from=2-7, to=2-8]
	\arrow[from=2-8, to=2-9]
\end{tikzcd}\]
and define their direct sum as 

\[\begin{tikzcd}[sep=tiny]
	{\xi \oplus \xi'\colon} & 0 & {B\oplus B} & {X_{n-1} \oplus X'_{n-1}} & \cdots & {X_0\oplus X_0} & {A \oplus A} & {0;}
	\arrow[from=1-2, to=1-3]
	\arrow[from=1-3, to=1-4]
	\arrow[from=1-4, to=1-5]
	\arrow[from=1-5, to=1-6]
	\arrow[from=1-6, to=1-7]
	\arrow[from=1-7, to=1-8]
\end{tikzcd}\]
then the sum $[\xi] + [\xi']$ is given by $[\left( \nabla_B \right)_* \left( \Delta_A \right)^* (\xi \oplus \xi')] $.

\begin{theorem}
	Baer sum makes the sets $\Ext^n_\C$ (for $n\geq 1$) into abelian groups, where the identity is given by the class of a split extension for $n=1$ and otherwise by the complex
	\[\begin{tikzcd}
		0 & B & B & 0 & \cdots & 0 & A & A & {0.}
		\arrow[from=1-1, to=1-2]
		\arrow["{\id_B}", from=1-2, to=1-3]
		\arrow[from=1-3, to=1-4]
		\arrow[from=1-4, to=1-5]
		\arrow[from=1-5, to=1-6]
		\arrow[from=1-6, to=1-7]
		\arrow["{\id_A}", from=1-7, to=1-8]
		\arrow[from=1-8, to=1-9]
	\end{tikzcd}\]
\end{theorem}









\section{Injective and projective objects}

\begin{definition}[Injective and projective objects]
	\label{def:inj-proj}
	Let $\mathcal{H}$ be a subclass of $\ob(\C)$. We say that an object $B$ is $\mathcal{H}$-\textbf{injective} if for every $A\in \mathcal{H}$ we have
	\[\Ext^1_\C(A,B) = 0.\]
	Dually we say that object $A$ is said to be $\mathcal{H}$-\textbf{projective} if for every $B\in \mathcal{H}$ we have
	\[\Ext^1_\C(A,B) = 0.\]
	If $\mathcal{H}= \ob(\C)$ the $\mathcal{H}$-injectives objects are simply called injective objects and the $\mathcal{H}$-projective objects are simply called projective objects. We will denote the class of injective objects of $\C$ as $\inj(\C)$ and the class of projective objects as $\proj(\C)$.
\end{definition}


From the definition it follows immediately that an object $B$ is $\mathcal{H}$-injective if and only if for every $A\in \mathcal{H}$ every sequence 
\[\begin{tikzcd}
	0 & B & X & A & 0
	\arrow[from=1-1, to=1-2]
	\arrow[from=1-2, to=1-3]
	\arrow[from=1-3, to=1-4]
	\arrow[from=1-4, to=1-5]
\end{tikzcd}\]
splits. The dual result is similar.


\begin{proposition}
	An object $P$ is projective if and only if for every epimorphism $e\colon E\longrightarrow X$ and every morphism $f\colon P \longrightarrow X$ there exists a morphism $\bar{f} \colon P \longrightarrow E$ such that 
	\[\begin{tikzcd}
		& E \\
		P & X
		\arrow["e", from=1-2, to=2-2]
		\arrow["{\bar{f}}", dashed, from=2-1, to=1-2]
		\arrow["f"', from=2-1, to=2-2]
	\end{tikzcd}\]
	commutes.
\end{proposition}

\begin{proof}
	($\Rightarrow $) Let $P$ be a projective object and let $e\colon E\longrightarrow X$ be an epimorphism and $f\colon P \longrightarrow X$ be a morphism; by taking a pullback
	\[\begin{tikzcd}
		{P\times_X E} & E \\
		P & {X,}
		\arrow["f'", from=1-1, to=1-2]
		\arrow["{e'}"', from=1-1, to=2-1]
		\arrow["e", from=1-2, to=2-2]
		\arrow["f"', from=2-1, to=2-2]
	\end{tikzcd}\]
	we get an epimorphism $e'\colon {P\times_X E} \longrightarrow P$ (see proposition \ref{prop:epi-pull}). By considering the exact sequence associated to it
	\[\begin{tikzcd}
		0 & {\ker(e')} & {P\times_X E} & P & 0
		\arrow[from=1-1, to=1-2]
		\arrow[from=1-2, to=1-3]
		\arrow["{e'}", from=1-3, to=1-4]
		\arrow["g"', curve={height=-24pt}, from=1-4, to=1-3]
		\arrow[from=1-4, to=1-5]
	\end{tikzcd}\]
	we get a section $g\colon P \longrightarrow {P\times_X E}$ , then $\bar{f} = f'g$.\\
	($\Leftarrow $) We are given a SES
	\[\begin{tikzcd}
		0 & A & B & P & 0
		\arrow[from=1-1, to=1-2]
		\arrow[from=1-2, to=1-3]
		\arrow["q", from=1-3, to=1-4]
		\arrow[from=1-4, to=1-5]
	\end{tikzcd}\]
	and we want to find a section of $q$. The necessary morphism is found by noticing that $q$ must be epic and applying the hypothesis to the following diagram:
	\[\begin{tikzcd}
		& B \\
		P & {P.}
		\arrow["q", from=1-2, to=2-2]
		\arrow[dashed, from=2-1, to=1-2]
		\arrow["{\id_P}"', from=2-1, to=2-2]
	\end{tikzcd}\]
\end{proof}

The dual result is also true.

\begin{proposition}
	An object $I$ is injective if and only if for every monomorphism $e \colon X\longrightarrow E$ and every morphism $f\colon X \longrightarrow I$ there exists a morphism $\bar{f} \colon P \longrightarrow E$ such that 
	\[\begin{tikzcd}
		X & I \\
		E
		\arrow["f", from=1-1, to=1-2]
		\arrow["e"', from=1-1, to=2-1]
		\arrow["{\bar f}"', dashed, from=2-1, to=1-2]
	\end{tikzcd}\]
	commutes.
\end{proposition}

\section{Exact functors}
\textbf{DECIDERE COSA SCRIVERE}






\section{Grothendieck Categories}




\begin{definition}
	Any abelian category $\C$ can also satisfy the following axioms:	
	\begin{itemize}
  		\item[(AB3)] $\C$ has all (small) coproducts (and therefore all small colimits);
    	\item[(AB4)] $\C$ satisfies (AB3) and the coproduct of monomorphisms is a monomorphism; 
    	\item[(AB5)] $\C$ satisfies (AB3) and filtered colimits in $\C$ are exact. 
	\end{itemize}
\end{definition}


\begin{definition}
	A \textbf{generator} $\mathcal{G}$ of a category $\C$ is an object $G$ which satisfies the following property: for every pair of distinct morphisms $f,g: A \longrightarrow B$, there is a morphism $s\colon G \longrightarrow A$ such that $fs\neq gs$. A \textbf{family of generators} is a nonempty set $\G=\{G_i\}_i$ which satisfies the following property: for every pair of distinct morphisms $f,g: A \longrightarrow B$, there is an object $G_i  \in \G$ and a morphism $s\colon G_i \longrightarrow A$ such that $fs\neq gs$.
\end{definition}

\begin{remark}
	It is easy to see that an object $G$ is a generator if and only if the functor $\Hom(G, {-})$ is faithful, therefore for any generator $G$ we have that $\Hom(G, X) = 0$ implies $X=0$. In an abelian category there is a slightly more interesting equivalent characterization: an object $G$ is a generator if and only if $\Hom(G, f) = f_*$ being surjective implies that $f$ is epic; similarly $\G$ is a family of generators if and only if $\Hom(G, f) = f_*$ being surjective for all $G \in \G$ implies that $f$ is epic. If we are working in an (AB3) category, the existence of a family of generators implies the existence of a generator: we only need to set
	\[G = \bigoplus_{G_i \in \G} G_i.\]
\end{remark}


\begin{definition}[Grothendieck category]
	An abelian category $\C$ is called a \textbf{Grothendieck} \textbf{category} if it satisfies (AB5) and possesses a generator.
\end{definition}

\begin{theorem}
	Every Grothendieck category has all (small) limits.
\end{theorem}


\begin{theorem}
	Every Grothendieck category has enough injectives.
\end{theorem}
\begin{proof}
	See \cite{Tohoku}.
\end{proof}

\begin{example}
	Modules/sheaves.
	\textbf{DECIDERE COSA SCRIVERE}
\end{example}


\textbf{FUNTORI DERIVATI STANDARD}