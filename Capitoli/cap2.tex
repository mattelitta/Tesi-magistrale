
\chapter{Abelian categories}


This first chapter will serve as an introduction to the basics of Abelian Categories, first introduced by Alexander Grothendieck in his seminal "Tohoku Paper" \cite{Tohoku}.
We will begin with the basic definitions.

\section{Basic definitions}
Let $\C$ be a category.
Let $u:A\longrightarrow B$ and $u':A'\longrightarrow B$ be monomorphisms in $\text{Ar}\: \C$, we say that $u$ majorizes $u'$, written $u \leq u'$, if there exists a morphism $v:A \longrightarrow A'$ such that $u=u'\circ v$. This relation defines a preorder on the class of arrows into $B$, from which we can extract the equivalence relation $u \cong v$ if and only if $u \leq v$ and $v \leq u$. We will not worry about size issues.



\begin{definition}[Subobject]
	Let $B \in \ob \: \C$, we define a subobject of $B$ to be an equivalence class of monomorphisms into $B$.
\end{definition}

We can now discuss the operation we can take between subobjects; every construction that we will describe is not dependent on the representative of the equivalence class of monomorphisms.

\begin{definition}[Intersection and union]
	Let $A \longrightarrow X$ and $B\longrightarrow X$ be subobjects, we define the intersection $A \cap B$ as the following pullback (when it exists):

	\[\begin{tikzcd}
		& {A\cap B} \\
		A && B \\
		& X.
		\arrow[dashed, from=1-2, to=2-1]
		\arrow[dashed, from=1-2, to=2-3]
		\arrow[dashed, from=1-2, to=3-2]
		\arrow[hook, from=2-1, to=3-2]
		\arrow[hook', from=2-3, to=3-2]
	\end{tikzcd}\]
	We can also define the sum $A+B$ of two subobjects as the following pushout (when it exists):
	\[\begin{tikzcd}
		& {A\cap B} \\
		A && B \\
		& {A+B} \\
		& X.
		\arrow[from=1-2, to=2-1]
		\arrow[from=1-2, to=2-3]
		\arrow[dashed, from=2-1, to=3-2]
		\arrow[from=2-1, to=4-2]
		\arrow[dashed, from=2-3, to=3-2]
		\arrow[from=2-3, to=4-2]
		\arrow[dashed, from=3-2, to=4-2]
	\end{tikzcd}\]

\end{definition}


\begin{remark}
	The preorder relation on monomorphisms induces a partial order on the subobjects of a specific object; the intersection is the meet of this poset, while the sum is the join.
\end{remark}

We can dualize all the previous constructions, to obtain the following definition.

\begin{definition}[Quotient]
	Let $B \in \ob \: \C$, we define a subobject of $B$ to be an equivalence class of epimorphisms from $B$.
\end{definition}


\section{Additive categories}


\begin{definition}[Additive category]
    A \textbf{additive category} is a (locally small) category $\mathcal{C}$ such that:
    \begin{itemize}
        \item for every couple of objects $A,B \in \ob \: \C$ the set of arrows $\Hom_\C(A,B)$ is an abelian group such that composition distributes over its group operation;
        \item $\C$ has binary products.
        \item $\C$ has a zero object, that is an object that is both initial and terminal.
    \end{itemize} 
    We will refer to the identity element of $\Hom_\C(A,B)$ as the $0$-map.
\end{definition}

We will now let $\C$ be an additive category.

\begin{remark}
	From the distributivity of composition we deduce that composition must be bilinear; therefore for every triplet of objects $A,B,C$ we get a morphism of abelian groups
	\begin{align}
		\Hom_\C(A,B) \otimes \Hom_\C(B,C) & \longrightarrow \Hom_\C(A, C) \\
		f \otimes g & \mapsto gf.
	\end{align}
	An immediate consequence of this fact is that the composition of $0$-maps is the $0$-map.
\end{remark}


\begin{proposition}
	\label{prop:mono-equiv}
	Let $f$ be a morphism in $\C$, the following are equivalents:
	\begin{itemize}
		\item $f$ is monic;
		\item $fg = 0$ implies $g=0$ for every composable $g$.
	\end{itemize}
\end{proposition}


\begin{proof}
	Recall that the monomorphisms between two objects $A,B$ are exactly the maps $f$ such that for every morphism $h:C \longrightarrow B$ we have at most one map $g:C \longrightarrow B$ such that the following diagram commutes:

	\[\begin{tikzcd}
		A & B \\
		C.
		\arrow["f", hook, from=1-1, to=1-2]
		\arrow["g", from=2-1, to=1-1]
		\arrow["h"', from=2-1, to=1-2]
	\end{tikzcd}\]
	Let $g_1,g_2$ be two such maps, then the commutativity of the diagram above is equivalent to requiring that $fg_1 = h = fg_2$, which implies $f(g_1 - g_2) = 0$. Given that $h$ can vary between al maps that factor through $f$, we can set $g = g_1 - g_2$ and get the thesis.

\end{proof}

The dual of the above proposition is also true.

\begin{proposition}
	Let $f$ be a morphism in $\C$, the following are equivalents:
	\begin{itemize}
		\item $f$ is epic;
		\item $gf = 0$ implies $g=0$ for every composable $g$.
	\end{itemize}
\end{proposition}


Let us now consider two objects $A,B$ in $\C$, we must have a (unique) terminal map $A \longrightarrow 0$ and a (unique) initial map $0 \longrightarrow B$; by uniqueness they must both be $0$-maps, so their composition is the zero map. We have shown that the $0$-map is the unique map that factors through the zero object.



We can easily check that every binary product $A \times B$ is also a binary coproduct, where the canonical maps $i_A$, $i_B$ are given by $(id_A, 0)$ and $(0, id_B)$.

\[\begin{tikzcd}
	& A \\
	& {A\times B} \\
	A && B
	\arrow["{i_A}", dashed, from=1-2, to=2-2]
	\arrow["{id_A}"', from=1-2, to=3-1]
	\arrow["0", from=1-2, to=3-3]
	\arrow["{\pi_A}", from=2-2, to=3-1]
	\arrow["{\pi_B}"', from=2-2, to=3-3]
\end{tikzcd}\]

We will denote the object that is both the product and the coproduct with the direct sum symbol, as an example we have $A\times B \cong A \amalg B \cong A \oplus B$.

\begin{definition}[Kernel and cokernel]
	Let $f : A\longrightarrow B$ be an arrow of $\C$, the kernel of $f$ is the equalizer 
	\[\begin{tikzcd}
	Ker(f) & A & B, & {}
	\arrow["0"', shift right=1, from=1-2, to=1-3]
	\arrow["f", shift left=1, from=1-2, to=1-3]
	\arrow[from=1-1, to=1-2]
	\end{tikzcd}\]
	while the cokernel is the coequalizer of the diagram
	\[\begin{tikzcd}
	 A & B & Coker(f). & {}
	\arrow["0"', shift right=1, from=1-1, to=1-2]
	\arrow["f", shift left=1, from=1-1, to=1-2]
	\arrow[from=1-2, to=1-3]
	\end{tikzcd}\]
	We will often refer to both the object and the arrow of the coequalizer as the kernel. We also define the image of $f$, denoted $Im f$, as $Ker(Coker(f))$ and the coimage of $f$, denoted $Coim(f)$, as $Coker(Ker(f))$.

\end{definition}

\begin{remark}
	Due to the fact that every equalizer is monic (and every coequalizer is epic), it is immediate to see that the kernel of a morphism is a subobject, while the cokernel is a quotient.
\end{remark}

Applying the relevant universal properties it is easy to see that there exist a canonical morphism $Coim(f) \longrightarrow Im(f)$ such that the following diagram commutes
\[\begin{tikzcd}
	{Ker(f)} & A && B & {Coker(f)} \\
	& {Coim(f)} && {Im(f).}
	\arrow[from=1-1, to=1-2]
	\arrow["f", from=1-2, to=1-4]
	\arrow[from=1-2, to=2-2]
	\arrow[from=1-4, to=1-5]
	\arrow[from=2-2, to=2-4]
	\arrow[from=2-4, to=1-4]
\end{tikzcd}\]

\begin{definition}[Additive functor]
	An \textbf{additive functor} $F: \C \longrightarrow \mathcal{D}$ is a functor between additive categories such that for every pair of objects $A,B \in \ob \: \C$ the induced map
	\[F_{A,B}:\Hom_\C(A,B) \longrightarrow \Hom_{\mathcal{D}}(FA,FB)\] 
	is a group homomorphism.
\end{definition}





\section{Abelian Categories}

\begin{definition}[Abelian category]
	An abelian category is an additive category $\C$ that satisfies the following axioms:
	\begin{itemize}
  		\item[(AB1)] any morphism admits a kernel and a cokernel;
    	\item[(AB2)] for every morphism $f:A\longrightarrow B$, the canonical map $Coim(f) \longrightarrow Im(f)$ is an isomorphism.
	\end{itemize}
\end{definition}

The existence of the kernel and the cokernel is remarkably powerful: the equalizer of $f$ and $g$ is exactly the kernel of $f-g$, while their coequalizer is its cokernel; therefore in any abelian category equalizers and coequalizers must exist. It is a well known result that it is possible to construct a finite limit using equalizers and finite products, thus every abelian category has finite limits and (dually) finite colimits. One consequence of this result is the fact that we can always take the intersection and the sum of subobjects, therefore the subobjects of a given object form a lattice where intersection and sum are meet and join.

\begin{remark}
	Let $f$ be a monomorphism, every maps that equalized $f$ and the $0$-map must necessarily be the $0$-map itself because of proposition \ref{prop:mono-equiv}; we also know that the $0$-map factors through $0$, therefore $0$ must be the kernel of $f$. Similarly the cokernel of an epimorphism must be $0$. It is also immediate that the kernel of $0 \longrightarrow B$ is $B$ (and the identity map), just like the cokernel of $A \longrightarrow 0$ is $A$ (and the identity map).
\end{remark}


From the previous remark follows that the coimage of the monomorphism $f: A \longrightarrow B $ is exactly $A$ (or more precisely the identity map $A \longrightarrow A$), therefore by applying (AB2) we deduce every monomorphism is its image and therefore it's a kernel.

\[\begin{tikzcd}
	0 & A & B & {coker(f)} \\
	& A & {Im(f)}
	\arrow[from=1-1, to=1-2]
	\arrow[from=1-2, to=1-3]
	\arrow["{id_A}"', from=1-2, to=2-2]
	\arrow[from=1-3, to=1-4]
	\arrow["\sim", from=2-2, to=2-3]
	\arrow[from=2-3, to=1-3]
\end{tikzcd}\]
Dually every epimorphism is its coimage and therefore it's a cokernel.



\section{Exact sequences}

\textbf{ROBA SULLE SUCCESSIONI ESATTE}
\newline


\begin{definition}
	Any abelian category $\C$ can also satisfy the following axioms:	
	\begin{itemize}
  		\item[(AB3)] $\C$ has all (small) coproducts (and therefore all small colimits);
    	\item[(AB4)] $\C$ satisfies (AB3) and the coproduct of monomorphisms is a monomorphism; 
    	\item[(AB5)] filtered colimits in $\C$ are exact. 
	\end{itemize}
\end{definition}

\section{Grothendieck Categories}



