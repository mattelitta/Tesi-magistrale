
\chapter{Abelian categories}


This first chapter will serve as an introduction to the basics of Abelian Categories, first introduced by Alexander Grothendieck in his seminal "Tohoku Paper" \cite{Tohoku}.
We will begin with the basic definitions.

\section{Basic definitions}
Let $\C$ be a category.
Let $u:A\longrightarrow B$ and $u':A'\longrightarrow B$ be monomorphisms in $\text{Ar}\: \C$, we say that $u$ majorizes $u'$, written $u \leq u'$, if there exists a morphism $v:A \longrightarrow A'$ such that $u=u'\circ v$. This relation defines a preorder on the class of arrows into $B$, from which we can extract the equivalence relation $u \cong v$ if and only if $u \leq v$ and $v \leq u$. We will generally not worry about size issues when taking quotients.



\begin{definition}[Subobject]
	Let $B \in \ob \: \C$, we define a \textbf{subobject} of $B$ to be an equivalence class of monomorphisms into $B$.
\end{definition}

We can now discuss the operation we can take between subobjects; every construction that we will describe is not dependent on the representative of the equivalence class of monomorphisms.

\begin{definition}[Intersection and union]
	Let $A \longrightarrow X$ and $B\longrightarrow X$ be subobjects, we define the \textbf{intersection} $A \cap B$ as the following pullback (when it exists):

	\[\begin{tikzcd}
		& {A\cap B} \\
		A && B \\
		& X.
		\arrow[dashed, from=1-2, to=2-1]
		\arrow[dashed, from=1-2, to=2-3]
		\arrow[dashed, from=1-2, to=3-2]
		\arrow[hook, from=2-1, to=3-2]
		\arrow[hook', from=2-3, to=3-2]
	\end{tikzcd}\]
	We can also define the \textbf{sum} $A+B$ of two subobjects as the following pushout (when it exists):
	\[\begin{tikzcd}
		& {A\cap B} \\
		A && B \\
		& {A+B} \\
		& {X.}
		\arrow[from=1-2, to=2-1]
		\arrow[from=1-2, to=2-3]
		\arrow[dashed, from=2-1, to=3-2]
		\arrow[from=2-1, to=4-2]
		\arrow[dashed, from=2-3, to=3-2]
		\arrow[from=2-3, to=4-2]
		\arrow[dashed, from=3-2, to=4-2]
	\end{tikzcd}\]

\end{definition}


\begin{remark}
	The preorder relation on monomorphisms induces a partial order on the subobjects of a specific object; the intersection is the meet of this poset, while the sum is the join.
\end{remark}

We can dualize all the previous constructions, to obtain the following definition.

\begin{definition}[Quotient]
	Let $B \in \ob \: \C$, we define a \textbf{quotient} of $B$ to be an equivalence class of epimorphisms from $B$.
\end{definition}


\section{Additive categories}

We start this section introducing additive categories and functors between them.

\begin{definition}[Additive category]
    A \textbf{additive category} is a (locally small) category $\mathcal{C}$ such that:
    \begin{itemize}
        \item for every couple of objects $A,B \in \ob \: \C$ the set of arrows $\Hom_\C(A,B)$ is an abelian group such that composition distributes over its group operation;
        \item $\C$ has binary products.
        \item $\C$ has a zero object, that is an object that is both initial and terminal.
    \end{itemize} 
    We will refer to the identity element of $\Hom_\C(A,B)$ as the $0$-map.
\end{definition}


\begin{definition}[Additive functor]
	An \textbf{additive functor} $F: \C \longrightarrow \mathcal{D}$ is a functor between additive categories such that for every pair of objects $A,B \in \ob \: \C$ the induced map
	\[F_{A,B}:\Hom_\C(A,B) \longrightarrow \Hom_{\mathcal{D}}(FA,FB)\] 
	is a group homomorphism.
\end{definition}


We will now let $\C$ be an additive category.

\begin{remark}
	From the distributivity of composition we deduce that composition must be bilinear; therefore for every triplet of objects $A,B,C$ we get a morphism of abelian groups
	\begin{align}
		\Hom_\C(A,B) \otimes \Hom_\C(B,C) & \longrightarrow \Hom_\C(A, C) \\
		f \otimes g & \mapsto gf.
	\end{align}
	An immediate consequence of this fact is that the composition of $0$-maps is the $0$-map.
\end{remark}


\begin{proposition}
	\label{prop:mono-equiv}
	Let $f$ be a morphism in $\C$, the following are equivalents:
	\begin{itemize}
		\item $f$ is monic;
		\item $fg = 0$ implies $g=0$ for every composable $g$.
	\end{itemize}
\end{proposition}


\begin{proof}
	Recall that the monomorphisms between two objects $A,B$ are exactly the maps $f$ such that for every morphism $h:C \longrightarrow B$ we have at most one map $g:C \longrightarrow B$ such that the following diagram commutes:

	\[\begin{tikzcd}
		A & B \\
		C.
		\arrow["f", hook, from=1-1, to=1-2]
		\arrow["g", from=2-1, to=1-1]
		\arrow["h"', from=2-1, to=1-2]
	\end{tikzcd}\]
	Let $g_1,g_2$ be two such maps, then the commutativity of the diagram above is equivalent to requiring that $fg_1 = h = fg_2$, which implies $f(g_1 - g_2) = 0$. Given that $h$ can vary between al maps that factor through $f$, we can set $g = g_1 - g_2$ and get the thesis.

\end{proof}

The dual of the above proposition is also true.

\begin{proposition}
	Let $f$ be a morphism in $\C$, the following are equivalents:
	\begin{itemize}
		\item $f$ is epic;
		\item $gf = 0$ implies $g=0$ for every composable $g$.
	\end{itemize}
\end{proposition}


Let us now consider two objects $A,B$ in $\C$, we must have a (unique) terminal map $A \longrightarrow 0$ and a (unique) initial map $0 \longrightarrow B$; by uniqueness they must both be $0$-maps, so their composition is the zero map. We have shown that the $0$-map is the unique map that factors through the zero object.



We can easily check that every binary product $A \times B$ is also a binary coproduct, where the canonical maps $i_1$, $i_2$ are given by $(\id_A, 0)$ and $(0, \id_B)$.

\[\begin{tikzcd}
	& A \\
	& {A\times B} \\
	A && B
	\arrow["{i_1}", dashed, from=1-2, to=2-2]
	\arrow["{id_A}"', from=1-2, to=3-1]
	\arrow["0", from=1-2, to=3-3]
	\arrow["{\pi_A}", from=2-2, to=3-1]
	\arrow["{\pi_B}"', from=2-2, to=3-3]
\end{tikzcd}\]

We will denote the object that is both the product and the coproduct with the direct sum symbol: we have $A\times B \cong A \amalg B \cong A \oplus B$, along all the standard relationships between the canonical maps.

\[\begin{tikzcd}
	A && B && A && B \\
	& {A \oplus B} &&&& {A \oplus B} \\
	A && B && B && A
	\arrow["{i_1}"', from=1-1, to=2-2]
	\arrow["{\id_A}", from=1-1, to=3-1]
	\arrow["{i_2}", from=1-3, to=2-2]
	\arrow["{\id_B}", from=1-3, to=3-3]
	\arrow["{i_1}"', from=1-5, to=2-6]
	\arrow["0", from=1-5, to=3-5]
	\arrow["{i_2}", from=1-7, to=2-6]
	\arrow["0", from=1-7, to=3-7]
	\arrow["{\pi_1}", from=2-2, to=3-1]
	\arrow["{\pi_2}"', from=2-2, to=3-3]
	\arrow["{\pi_2}", from=2-6, to=3-5]
	\arrow["{\pi_1}"', from=2-6, to=3-7]
\end{tikzcd}\]

We also denote the maps produced from $f,g$ (resp. $f',g'$) by the universal property of the (co)product as $(f,g)$ (resp. $f+g$), taking care to note that $f+g$ has not (yet) any relation with the sum in $\Hom(A\oplus B, B)$. 

\[\begin{tikzcd}
	& X &&& A & {A\oplus B} & B \\
	A & {A \oplus B} & B &&& Y
	\arrow["f"', from=1-2, to=2-1]
	\arrow["{(f,g)}", dashed, from=1-2, to=2-2]
	\arrow["g", from=1-2, to=2-3]
	\arrow["{i_1}", from=1-5, to=1-6]
	\arrow["{f'}"', from=1-5, to=2-6]
	\arrow["{f+g}", dashed, from=1-6, to=2-6]
	\arrow["{i_2}"', from=1-7, to=1-6]
	\arrow["{g'}", from=1-7, to=2-6]
	\arrow["{\pi_1}", from=2-2, to=2-1]
	\arrow["{\pi_2}"', from=2-2, to=2-3]
\end{tikzcd}\]

We will also denote the diagonal map $(\id_A, \id_A)$ as $\Delta_A \colon A \longrightarrow A \otimes A$ and the codiagonal map $id_B + id_B$ as $\nabla_B \colon B\oplus B \longrightarrow B$.\\
With an analogous construction we get direct sum of maps $f \oplus g$, with all the relevant properties.


\[\begin{tikzcd}
	& A & B \\
	{A\oplus A'} &&& {B\oplus B'} \\
	& {A'} & {B'}
	\arrow["f", from=1-2, to=1-3]
	\arrow[from=1-3, to=2-4]
	\arrow["{\pi_1}", from=2-1, to=1-2]
	\arrow["{f \oplus g}", from=2-1, to=2-4]
	\arrow["{\pi_2}"', from=2-1, to=3-2]
	\arrow["g"', from=3-2, to=3-3]
	\arrow[from=3-3, to=2-4]
\end{tikzcd}\]


\[\begin{tikzcd}
	A && {} & B \\
	& {A\oplus A'} & {B\oplus B'} \\
	{A'} &&& {B'}
	\arrow["f", from=1-1, to=1-4]
	\arrow["{i_1}"', from=1-1, to=2-2]
	\arrow["{f\oplus g}", from=2-2, to=2-3]
	\arrow["{\pi_1}"', from=2-3, to=1-4]
	\arrow["{\pi_2}", from=2-3, to=3-4]
	\arrow["{i_2}", from=3-1, to=2-2]
	\arrow["g", from=3-1, to=3-4]
\end{tikzcd}\]




\begin{remark}
	Let us consider the map $\pi_1+\pi_2\colon B\oplus B \longrightarrow B$ given by the additive structure on $\Hom(B\oplus B, B)$; we have that $(\pi_1 +\pi_2)i_1 = \pi_1 i_1 + \pi_2 i_1 = \id_B + 0 =\id_B $ and that $(\pi_1 +\pi_2)i_2 =\id_B $, therefore we can write $\nabla_B = \pi_1+\pi_2$. A consequence of this result is that, given two maps $f,g \colon A \longrightarrow B$, we have that
	\[f+g = \nabla_A (f \oplus g) \Delta_B.\]
\end{remark}

\begin{definition}[Kernel and cokernel]
	Let $f : A\longrightarrow B$ be an arrow of $\C$, the \textbf{kernel} of $f$ is the equalizer 
	\[\begin{tikzcd}
	Ker(f) & A & B, & {}
	\arrow["0"', shift right=1, from=1-2, to=1-3]
	\arrow["f", shift left=1, from=1-2, to=1-3]
	\arrow[from=1-1, to=1-2]
	\end{tikzcd}\]
	while the \textbf{cokernel} is the coequalizer of the diagram
	\[\begin{tikzcd}
	 A & B & Coker(f). & {}
	\arrow["0"', shift right=1, from=1-1, to=1-2]
	\arrow["f", shift left=1, from=1-1, to=1-2]
	\arrow[from=1-2, to=1-3]
	\end{tikzcd}\]
	We will often refer to both the object and the arrow of the coequalizer as the kernel. We also define the image of $f$, denoted $Im f$, as $Ker(Coker(f))$ and the coimage of $f$, denoted $Coim(f)$, as $Coker(Ker(f))$.

\end{definition}


\begin{remark}
	Due to the fact that every equalizer is monic (and every coequalizer is epic), it is immediate to see that the kernel of a morphism is a subobject, while the cokernel is a quotient. This also implies that the image is a subobject while the coimage is a quotient.
\end{remark}

Applying the relevant universal properties it is easy to see that there exist a canonical morphism $Coim(f) \longrightarrow Im(f)$ such that the following diagram commutes
\[\begin{tikzcd}
	{Ker(f)} & A && B & {Coker(f)} \\
	& {Coim(f)} && {Im(f).}
	\arrow[from=1-1, to=1-2]
	\arrow["f", from=1-2, to=1-4]
	\arrow[from=1-2, to=2-2]
	\arrow[from=1-4, to=1-5]
	\arrow[from=2-2, to=2-4]
	\arrow[from=2-4, to=1-4]
\end{tikzcd}\]






\section{Abelian Categories}

\begin{definition}[Abelian category]
	An \textbf{abelian category} is an additive category $\C$ that satisfies the following axioms:
	\begin{itemize}
  		\item[(AB1)] any morphism admits a kernel and a cokernel;
    	\item[(AB2)] for every morphism $f:A\longrightarrow B$, the canonical map $Coim(f) \longrightarrow Im(f)$ is an isomorphism.
	\end{itemize}
\end{definition}

The existence of the kernel and the cokernel is remarkably powerful: the equalizer of $f$ and $g$ is exactly the kernel of $f-g$, while their coequalizer is its cokernel; therefore in any abelian category equalizers and coequalizers must exist. It is a well known result that it is possible to construct a finite limit using equalizers and finite products, thus every abelian category has finite limits and (dually) finite colimits. One consequence of this result is the fact that we can always take the intersection and the sum of subobjects, therefore the subobjects of a given object form a lattice where intersection and sum are meet and join.

\begin{remark}
	Let $f$ be a monomorphism, every maps that equalized $f$ and the $0$-map must necessarily be the $0$-map itself because of proposition \ref{prop:mono-equiv}; we also know that the $0$-map factors through $0$, therefore $0$ must be the kernel of $f$. Similarly the cokernel of an epimorphism must be $0$. With the 
	 It is also immediate that the kernel of $0 \longrightarrow B$ is $B$ (and the identity map), just like the cokernel of $A \longrightarrow 0$ is $A$ (and the identity map).
\end{remark}


From the previous remark follows that the coimage of the monomorphism $f: A \longrightarrow B $ is exactly $A$ (or more precisely the identity map $A \longrightarrow A$), therefore by applying (AB2) we deduce every monomorphism is its image and therefore it's the kernel of a map.

\[\begin{tikzcd}
	0 & A & B & {coker(f)} \\
	& A & {Im(f)}
	\arrow[from=1-1, to=1-2]
	\arrow[from=1-2, to=1-3]
	\arrow["{\id_A}"', from=1-2, to=2-2]
	\arrow[from=1-3, to=1-4]
	\arrow["\sim", from=2-2, to=2-3]
	\arrow[from=2-3, to=1-3]
\end{tikzcd}\]
Dually every epimorphism is its coimage and therefore it's the cokernel of a map. \\
We continue by introducing one of the main object used in the study of abelian categories: exact sequences.

\begin{definition}[Exact sequence]
	Let $f\colon A\longrightarrow B$ and $g\colon B \longrightarrow C$ be morphisms in $\C$, we say that the sequence 
	\[\begin{tikzcd}
		A & B & C
		\arrow["f", from=1-1, to=1-2]
		\arrow["g", from=1-2, to=1-3]
	\end{tikzcd}\]
	is \textbf{exact} if $Im(f)$ and $Ker(g)$ are the same subobject of $B$. More generally if we have a succession of morphism 
	\[\begin{tikzcd}
		\cdots & {A_{k-1}} & {A_k} & {A_{k+1}} & \cdots
		\arrow[from=1-1, to=1-2]
		\arrow["{f_{k-1}}", from=1-2, to=1-3]
		\arrow["{f_k}", from=1-3, to=1-4]
		\arrow[from=1-4, to=1-5]
	\end{tikzcd}\]
	we say that it is exact at $A_k$ if $A_{k-1}\longrightarrow A_k \longrightarrow A_{k+1}$ is exact; we say that it is exact if it is exact at $A_k$ for every $k$. A \textbf{short exact sequence} is a sequence of the form
	\[\begin{tikzcd}
		0 & A & B & C & 0
		\arrow[from=1-1, to=1-2]
		\arrow[from=1-2, to=1-3]
		\arrow[from=1-3, to=1-4]
		\arrow[from=1-4, to=1-5]
	\end{tikzcd}\]
	that is exact at $A$, $B$ and $C$; we will often shorten the name short exact sequence as SES.
\end{definition}



\section{Exact functors}
\textbf{DECIDERE COSA SCRIVERE}




\section{Extensions}
In this section we will essentially follow \cite{Weibel_1994} and \cite{maclane2012homology}.
Let $\C$ be an abelian category, let $A,B$ be objects of $\C$.

\begin{definition}[Extension]
	An \textbf{extension} $\xi$ of $A$ and $B$ is a short exact sequence of the form
	\[\begin{tikzcd}
		{\xi\colon} & 0 & B & X & A & {0.}
		\arrow[from=1-2, to=1-3]
		\arrow[from=1-3, to=1-4]
		\arrow[from=1-4, to=1-5]
		\arrow[from=1-5, to=1-6]
	\end{tikzcd}\]
	More generally, for $n\geq 1$, an $n$-\textbf{extension} $\xi$ of $A$ and $B$ is an exact sequence 
	\[\begin{tikzcd}[sep=small]
		{\xi \colon} & 0 & B & {X_{n-1}} & {X_{n-2}} & \cdots & {X_0} & A & 0
		\arrow[from=1-2, to=1-3]
		\arrow[from=1-3, to=1-4]
		\arrow[from=1-4, to=1-5]
		\arrow[from=1-5, to=1-6]
		\arrow[from=1-6, to=1-7]
		\arrow[from=1-7, to=1-8]
		\arrow[from=1-8, to=1-9]
	\end{tikzcd}\]
\end{definition}

We can now define an equivalence relation on the class of $n$-extensions as follows: we say that the extension $\xi$ and $\xi'$ are equivalent if there is a commutative diagram

\[\begin{tikzcd}[sep=small]
	{\xi \colon} & 0 & B & {X_{n-1}} & {X_{n-2}} & \cdots & {X_0} & A & 0 \\
	& 0 & B & {X''_{n-1}} & {X''_{n-2}} & \cdots & {X''_0} & A & {0} \\
	{\xi'\colon} & 0 & A & {X_{n-1}'} & {X_{n-2}'} & \cdots & {X'_0} & B & {0.}
	\arrow[from=1-2, to=1-3]
	\arrow[from=1-3, to=1-4]
	\arrow[from=1-4, to=1-5]
	\arrow[from=1-5, to=1-6]
	\arrow[from=1-6, to=1-7]
	\arrow[from=1-7, to=1-8]
	\arrow[from=1-8, to=1-9]
	\arrow[from=2-2, to=2-3]
	\arrow["{\id_B}", from=2-3, to=1-3]
	\arrow[from=2-3, to=2-4]
	\arrow["{\id_B}"', from=2-3, to=3-3]
	\arrow[from=2-4, to=1-4]
	\arrow[from=2-4, to=2-5]
	\arrow[from=2-4, to=3-4]
	\arrow[from=2-5, to=1-5]
	\arrow[from=2-5, to=2-6]
	\arrow[from=2-5, to=3-5]
	\arrow[from=2-6, to=2-7]
	\arrow[from=2-7, to=1-7]
	\arrow[from=2-7, to=2-8]
	\arrow[from=2-7, to=3-7]
	\arrow["{\id_A}", from=2-8, to=1-8]
	\arrow[from=2-8, to=2-9]
	\arrow["{\id_A}"', from=2-8, to=3-8]
	\arrow[from=3-2, to=3-3]
	\arrow[from=3-3, to=3-4]
	\arrow[from=3-4, to=3-5]
	\arrow[from=3-5, to=3-6]
	\arrow[from=3-6, to=3-7]
	\arrow[from=3-7, to=3-8]
	\arrow[from=3-8, to=3-9]
\end{tikzcd}\]

\begin{remark}
	If $n=1$, the relation is much simpler: in the following diagram both middle maps must necessarily be isomorphism, therefore we need only to look for a map $X \to X'$ that makes the following diagram commute.
	\[\begin{tikzcd}[sep=small]
		{\xi \colon} & 0 & B & X & A & 0 \\
		{\xi'\colon} & 0 & B & {X'} & A & 0
		\arrow[from=1-2, to=1-3]
		\arrow[from=1-3, to=1-4]
		\arrow[from=1-4, to=1-5]
		\arrow[from=1-5, to=1-6]
		\arrow[from=2-2, to=2-3]
		\arrow["{\id_B}", from=1-3, to=2-3]
		\arrow[from=2-3, to=2-4]
		\arrow[from=1-4, to=2-4]
		\arrow[from=2-4, to=2-5]
		\arrow["{\id_A}", from=1-5, to=2-5]
		\arrow[from=2-5, to=2-6]
	\end{tikzcd}\]

\end{remark}

\begin{definition}[Split extensions]
	We say that an extension of $A$ and $B$ is \textbf{split} if it is equivalent to
	\[\begin{tikzcd}
	0 & B & {B\oplus A} & A & 0.
	\arrow[from=1-1, to=1-2]
	\arrow["{i_A}", from=1-2, to=1-3]
	\arrow["{\pi_B}", from=1-3, to=1-4]
	\arrow[from=1-4, to=1-5]
\end{tikzcd}\]
\end{definition}

\begin{definition}[Ext]
	We define $\Ext^n_\C(A,B)$ to be quotient of the family of all $n$-extensions by the relation of equivalence. We will also define $\Ext^0_\C(A,B)$ as $\Hom_\C(A,B)$. 
\end{definition}
\begin{remark}
	$\Ext^n_\C(A,B)$ is not necessarily small (i.e. a set), see \cite[\href{https://stacks.math.columbia.edu/tag/07JS}{Section 07JS}]{stacks-project}, but it will be in the relevant cases (Grothendieck categories). From now on, in this section, we will assume it is a set. 
\end{remark}

We will show that $\Ext$ is actually a functor from $\C\op \times \C$ to $\cat{Set}$, in the case $n=1$ (for the general case see \cite{maclane2012homology}). 
If we are given an extension 
\[0 \longrightarrow B \longrightarrow X \longrightarrow A \longrightarrow 0 \]
and $h\colon B\longrightarrow B'$ we can take the pushout ${B'\times_B X}$; then, applying the relevant universal property to the maps $0\colon B' \longrightarrow A$ and $g \colon X \longrightarrow A$, we get a commuting diagrams with exact lines:


\[\begin{tikzcd}
	0 & B & X & A & 0 \\
	0 & {B'} & {B'\times_B X} & A & {0.}
	\arrow[from=1-1, to=1-2]
	\arrow["f", from=1-2, to=1-3]
	\arrow["h", from=1-2, to=2-2]
	\arrow["g", from=1-3, to=1-4]
	\arrow[from=1-3, to=2-3]
	\arrow[from=1-4, to=1-5]
	\arrow["{\id_A}", from=1-4, to=2-4]
	\arrow[from=2-1, to=2-2]
	\arrow["{f'}", from=2-2, to=2-3]
	\arrow["{g'}", from=2-3, to=2-4]
	\arrow[from=2-4, to=2-5]
\end{tikzcd}\]
Therefore, by passing to the quotient, we get a morphism of sets \[h_{*}\colon \Ext^1(A,B) \longrightarrow \Ext(A,B').\] 
Similarly, given a map $k:A'\longrightarrow A$, we get a function 
\[k_* \colon \Ext^1_\C(A, B) \longrightarrow \Ext^1_\C(A',B).\] 

We will now introduce an operation, called the \textit{Baer sum}, on $\Ext^n_\C(A,B)$: given two elements in $\Ext^n_\C(A,B)$ with representatives

\[\begin{tikzcd}[sep=small]
	{\xi \colon} & 0 & B & {X_{n-1}} & {X_{n-2}} & \cdots & {X_0} & A & 0 \\
	{\xi' \colon} & 0 & B & {X'_{n-1}} & {X'_{n-2}} & \cdots & {X'_0} & A & 0
	\arrow[from=1-2, to=1-3]
	\arrow[from=1-3, to=1-4]
	\arrow[from=1-4, to=1-5]
	\arrow[from=1-5, to=1-6]
	\arrow[from=1-6, to=1-7]
	\arrow[from=1-7, to=1-8]
	\arrow[from=1-8, to=1-9]
	\arrow[from=2-2, to=2-3]
	\arrow[from=2-3, to=2-4]
	\arrow[from=2-4, to=2-5]
	\arrow[from=2-5, to=2-6]
	\arrow[from=2-6, to=2-7]
	\arrow[from=2-7, to=2-8]
	\arrow[from=2-8, to=2-9]
\end{tikzcd}\]
and define their direct sum as 

\[\begin{tikzcd}[sep=tiny]
	{\xi \oplus \xi'\colon} & 0 & {B\oplus B} & {X_{n-1} \oplus X'_{n-1}} & \cdots & {X_0\oplus X_0} & {A \oplus A} & {0;}
	\arrow[from=1-2, to=1-3]
	\arrow[from=1-3, to=1-4]
	\arrow[from=1-4, to=1-5]
	\arrow[from=1-5, to=1-6]
	\arrow[from=1-6, to=1-7]
	\arrow[from=1-7, to=1-8]
\end{tikzcd}\]
then the sum $[\xi] + [\xi']$ is given by $[\left( \nabla_B \right)_* \left( \Delta_A \right)^* (\xi \oplus \xi')] $.

\begin{theorem}
	
\end{theorem}


\section{Grothendieck Categories}

\begin{definition}
	Any abelian category $\C$ can also satisfy the following axioms:	
	\begin{itemize}
  		\item[(AB3)] $\C$ has all (small) coproducts (and therefore all small colimits);
    	\item[(AB4)] $\C$ satisfies (AB3) and the coproduct of monomorphisms is a monomorphism; 
    	\item[(AB5)] filtered colimits in $\C$ are exact. 
	\end{itemize}
\end{definition}


\begin{definition}
	A generator $\mathcal{G}$ of a category $\C$ is a family of objects $\mathcal{G} \subseteq \ob \: \C$ which satisfies the following property: for every pair of distinct morphisms $f,g: A \longrightarrow B$ there is an object $G \in \mathcal{G}$ and a morphism $h: G \longrightarrow A$ such that $fh \neq gh$.
\end{definition}


\begin{definition}[Grothendieck category]
	A category $\C$ is called a Grothendieck category if it is an abelian category that satisfies (AB5) and possesses a generator.
\end{definition}







\section{Monoidal categories}


\textbf{DECIDERE SE/COSA SCRIVERE}