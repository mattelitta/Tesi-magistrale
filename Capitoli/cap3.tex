\chapter{Model categories}

We will introduce the concept of model categories and homotopy categories to define derived categories without recurring to the calculus of fractions. The main reference we will be using is \cite{dwyer1995homotopy}, while \cite{hovey2007model} is a very good expository paper.

\section{Basic definitions}
Let $\C$ be a category.

\begin{definition}[Lift]
    Given a commuting square in $\C$
    \[\begin{tikzcd}
	    A & X \\
	    B & {Y,}
	    \arrow[from=1-1, to=1-2]
	    \arrow["i", from=1-1, to=2-1]
	    \arrow["p", from=1-2, to=2-2]
	    \arrow[from=2-1, to=2-2]
    \end{tikzcd}\]
    a \textbf{lift} is a map $f\colon B \longrightarrow X$ such that
    \[\begin{tikzcd}
	    A & X \\
	    B & Y
	    \arrow[from=1-1, to=1-2]
	    \arrow["i", from=1-1, to=2-1]
	    \arrow["p", from=1-2, to=2-2]
	    \arrow["f", from=2-1, to=1-2]
	    \arrow[from=2-1, to=2-2]
    \end{tikzcd}\]
    commutes. In the case where a lift exists we will say that $i$ has the \textbf{left lifting property} (shortened as LLP) with respect to $p$; dually we will say that $p$ has the \textbf{right lifting property} (shortened as RLP) with respect to $i$.

\end{definition}


\begin{definition}[Retract]
    Let $X,Y$ be objects in $\C$, we say that $X$ is a retract of $Y$ if there exists arrows $f,g$ such that $gf = \id_X$.    
\end{definition}


\begin{remark}
	The case we are most interested in is when we are taking retracts in a category of morphisms, unraveling the definition we get the following: if $\C = \ar(\D)$ then $f$ is a retract of $g$ if and only if there exists $i,i',p,p'$ such that
    \[\begin{tikzcd}
	    X & Y & X \\
	    {X'} & {Y'} & {X'}
	    \arrow["i", from=1-1, to=1-2]
	    \arrow["{\id_X}"', curve={height=-18pt}, from=1-1, to=1-3]
	    \arrow["f", from=1-1, to=2-1]
	    \arrow["p", from=1-2, to=1-3]
	    \arrow["g", from=1-2, to=2-2]
	    \arrow["f", from=1-3, to=2-3]
	    \arrow["{i'}", from=2-1, to=2-2]
	    \arrow["{\id_{X'}}"', curve={height=18pt}, from=2-1, to=2-3]
	    \arrow["{p'}", from=2-2, to=2-3]
    \end{tikzcd}\]
	commutes.
\end{remark}


We will now state a lemma that we will often use in dealing with model categories.

\begin{lemma}[The retract argument]
	Let $f\colon A \longrightarrow C$ be an arrow of $\C$ that can be factored as $f=pi$ and suppose that $f$ has the LLP with respect to $p$. Then $f$ is a retract of $i$. Dually, if $f$ has the RLP with respect to $i$, it is a retract of $p$.
\end{lemma}
\begin{proof}
	We will only show the first case, the proof of the second case is dual to the one shown. Let us begin by observing that requiring that $f=pi$ is equivalent to having the following diagram commute:
	\[\begin{tikzcd}
		A & B \\
		C & {C.}
		\arrow["i"', from=1-1, to=1-2]
		\arrow["f", from=1-1, to=2-1]
		\arrow["p"', from=1-2, to=2-2]
		\arrow["{\id_C}", from=2-1, to=2-2]
	\end{tikzcd}\]
	Applying the LLP to the square and expanding the diagram we get
	\[\begin{tikzcd}
		A & A & A \\
		C & B & {C,}
		\arrow["{\id_A}"', from=1-1, to=1-2]
		\arrow["f", from=1-1, to=2-1]
		\arrow["{\id_A}"', from=1-2, to=1-3]
		\arrow["i"', from=1-2, to=2-2]
		\arrow["f"', from=1-3, to=2-3]
		\arrow["g", from=2-1, to=2-2]
		\arrow["p", from=2-2, to=2-3]
	\end{tikzcd}\]
	where $g$ is the map that we get from the LLP. This shows exactly the thesis.
\end{proof}

We can now introduce the main object that we will discuss.

Let us consider a triple $(W, \fibT, \cofT)$ of subclasses of $\ar(\C)$, called respectively weak equivalences ($\we$), fibrations ($\fib$) and cofibrations ($\cof$); we will name the class $W\cap \fibT $ as acyclical fibrations ($\afib$) and $W \cap \cofT$ as acyclical cofibrations ($\acof$).

\section{Model categories}

\begin{definition}[Model structure]
	They are said to form a \textbf{model structure} on $\C$ if they satisfy the following axioms:
	\begin{itemize}
		\item[(\textbf{MC1})] If $f$ and $g$ are composable morphisms of $\C$ and two out three between $f,g,gf$ are weak equavalences then also the this is.
		\item[(\textbf{MC2})] $W$ is closed under retractions.
		\item[(\textbf{MC3})] Trivial cofibrations have the LLP with respect to the fibrations, while the trivial fibrations have the RLP with the respect to the cofibrations.
		\item[(\textbf{MC4})] Every arrow $f$ admits both of the following factorizations: (1) $f=pi$ where $i$ is an acyclical cofibration and $p$ is a fibration, and (2) $f=pi$ where $i$ is a cofibration and $p$ is an acyclical fibration.
	\end{itemize}
\end{definition}

\begin{definition}[Model category]
	A \textbf{model category} $\C$ is a complete and cocomplete category along with a model structure on it. We will denote its weak equivalences as $W(\C)$, its fibrations as $\fibT(\C)$ and its cofibrations as $\cofT(\C)$.
\end{definition}

\begin{remark}
	Sometimes in the definition of model category it is required that the factorization be functorial; we will not make such demand.
\end{remark}

