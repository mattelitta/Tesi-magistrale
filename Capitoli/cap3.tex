\chapter{Model categories}

We will introduce the concept of model categories and homotopy categories to define derived categories without recurring to the calculus of fractions. The main reference we will be using is \cite{dwyer1995homotopy}, while \cite{hovey2007model} is a very good expository paper.

\section{Basic definitions}
Let $\C$ be a category.

\begin{definition}[Lift]
    Given a commuting square in $\C$
    \[\begin{tikzcd}
	    A & X \\
	    B & {Y,}
	    \arrow[from=1-1, to=1-2]
	    \arrow["i", from=1-1, to=2-1]
	    \arrow["p", from=1-2, to=2-2]
	    \arrow[from=2-1, to=2-2]
    \end{tikzcd}\]
    a \textbf{lift} is a map $f\colon B \longrightarrow X$ such that
    \[\begin{tikzcd}
	    A & X \\
	    B & Y
	    \arrow[from=1-1, to=1-2]
	    \arrow["i", from=1-1, to=2-1]
	    \arrow["p", from=1-2, to=2-2]
	    \arrow["f", from=2-1, to=1-2]
	    \arrow[from=2-1, to=2-2]
    \end{tikzcd}\]
    commutes. In the case where a lift exists we will say that $i$ has the \textbf{left lifting property} (shortened as LLP) with respect to $p$; dually we will say that $p$ has the \textbf{right lifting property} (shortened as RLP) with respect to $i$.

\end{definition}


\begin{definition}[Retract]
    Let $X,Y$ be objects in $\C$, we say that $X$ is a retract of $Y$ if there exists arrows $f,g$ such that $gf = \id_X$.    
\end{definition}


\begin{remark}
	The case we are most interested in is when we are taking retracts in a category of morphisms, unraveling the definition we get the following: if $\C = \ar(\D)$ then $f$ is a retract of $g$ if and only if there exists $i,i',p,p'$ such that
    \[\begin{tikzcd}
	    X & Y & X \\
	    {X'} & {Y'} & {X'}
	    \arrow["i", from=1-1, to=1-2]
	    \arrow["{\id_X}", curve={height=-18pt}, from=1-1, to=1-3]
	    \arrow["f", from=1-1, to=2-1]
	    \arrow["p", from=1-2, to=1-3]
	    \arrow["g", from=1-2, to=2-2]
	    \arrow["f", from=1-3, to=2-3]
	    \arrow["{i'}", from=2-1, to=2-2]
	    \arrow["{\id_{X'}}"', curve={height=18pt}, from=2-1, to=2-3]
	    \arrow["{p'}", from=2-2, to=2-3]
    \end{tikzcd}\]
	commutes.
\end{remark}


We will now state a lemma that we will often use in dealing with model categories.

\begin{lemma}[The retract argument]
	Let $f\colon A \longrightarrow C$ be an arrow of $\C$ that can be factored as $f=pi$ and suppose that $f$ has the LLP with respect to $p$. Then $f$ is a retract of $i$. Dually, if $f$ has the RLP with respect to $i$, it is a retract of $p$.
\end{lemma}
\begin{proof}
	We will only show the first case, the proof of the second case is dual to the one shown. Let us begin by observing that requiring that $f=pi$ is equivalent to having the following diagram commute:
	\[\begin{tikzcd}
		A & B \\
		C & {C.}
		\arrow["i"', from=1-1, to=1-2]
		\arrow["f", from=1-1, to=2-1]
		\arrow["p"', from=1-2, to=2-2]
		\arrow["{\id_C}", from=2-1, to=2-2]
	\end{tikzcd}\]
	Applying the LLP to the square and expanding the diagram we get
	\[\begin{tikzcd}
		A & A & A \\
		C & B & {C,}
		\arrow["{\id_A}"', from=1-1, to=1-2]
		\arrow["f", from=1-1, to=2-1]
		\arrow["{\id_A}"', from=1-2, to=1-3]
		\arrow["i"', from=1-2, to=2-2]
		\arrow["f"', from=1-3, to=2-3]
		\arrow["g", from=2-1, to=2-2]
		\arrow["p", from=2-2, to=2-3]
	\end{tikzcd}\]
	where $g$ is the map that we get from the LLP. This shows exactly the thesis.
\end{proof}

We can now introduce the main object that we will discuss.

Let us consider a triple $(W, \fibT, \cofT)$ of subclasses of $\ar(\C)$, called respectively weak equivalences ($\we$), fibrations ($\fib$) and cofibrations ($\cof$); we will name the class $W\cap \fibT $ as acyclic fibrations ($\afib$) and $W \cap \cofT$ as acyclic cofibrations ($\acof$).

\section{Model categories}

\begin{definition}[Model structure]
	They are said to form a \textbf{model structure} on $\C$ if they satisfy the following axioms:
	\begin{itemize}
		\item[(\textbf{MC1})] If $f$ and $g$ are composable morphisms of $\C$ and two out three between $f,g,gf$ are weak equavalences then also the this is.
		\item[(\textbf{MC2})] $W$, $\fibT$ and $\cofT$ are closed under retractions.
		\item[(\textbf{MC3})] Trivial cofibrations have the LLP with respect to the fibrations, while the trivial fibrations have the RLP with the respect to the cofibrations.
		\item[(\textbf{MC4})] Every arrow $f$ admits both of the following factorizations: (1) $f=pi$ where $i$ is an acyclic cofibration and $p$ is a fibration, and (2) $f=pi$ where $i$ is a cofibration and $p$ is an acyclic fibration.
	\end{itemize}
\end{definition}

\begin{definition}[Model category]
	A \textbf{model category} $\C$ is a complete and cocomplete category along with a model structure on it. We will denote its weak equivalences as $W(\C)$, its fibrations as $\fibT(\C)$ and its cofibrations as $\cofT(\C)$.
\end{definition}

\begin{remark}
	Sometimes in the definition of model category it is required that the factorization be functorial; we will not make such demand.
\end{remark}

\begin{remark}
	
\end{remark}

We will from now on use $\C$ to refer to a model category.

\begin{proposition}
	In a model category, the cofibrations (resp. trivial cofibrations) are exactly the maps that have the LLP with respect to all trivial fibrations (resp. fibrations). Dually the fibrations (resp. trivial fibrations) are exactly the maps that have the RLP with respect to all trivial cofibrations (resp. cofibrations).
\end{proposition}
\begin{proof}
	We will only show the characterization of cofibrations, the other are analogous. By (\textbf{MC3}) cofibrations must have such property, therefore we need only to show the converse: let $f: A\longrightarrow \C$ be a map that has the LLP with respect to all trivial fibrations, by applying (\textbf{MC4}) we can obtain a factorization $f=pi$ where $p$ is a trivial fibration and $i$ is a cofibration; by applying the retract argument and using (\textbf{MC2}) we conclude.
\end{proof}

\begin{corollary}
	Every isomorphism is both a trivial fibration and a trivial cofibration. This in particular is true for the identity maps.
\end{corollary}
\begin{proof}
	The corollary follows immediately from the proposition by using the inverse morphism to construct lifts.
\end{proof}

\begin{corollary}
	The class of weak equivalences, fibrations and cofibrations is closed under composition.
\end{corollary}
\begin{proof}
	The case of weak equivalences is a consequence of (\textbf{MC1}), we will prove the case of fibrations (the other is dual).
	Using the proposition, we need only to show that the composition of two fibrations $p$, $p'$ has the RLP with respect to any acyclic fibration $i$. To do so, it is sufficient to apply (\textbf{MC3}) twice as in the diagram below, first to get $f$ and then again in the new square to obtain $g$.
	\[\begin{tikzcd}
		A && X \\
		&& Y \\
		B && Z
		\arrow[from=1-1, to=1-3]
		\arrow["i", "\sim"', hook', from=1-1, to=3-1]
		\arrow["p", two heads, from=1-3, to=2-3]
		\arrow["{p'}", two heads, from=2-3, to=3-3]
		\arrow["g", dashed, from=3-1, to=1-3]
		\arrow["f"', dashed, from=3-1, to=2-3]
		\arrow[from=3-1, to=3-3]
	\end{tikzcd}\]
\end{proof}

\begin{corollary}
	Being a fibration (resp. cofibration) is a property that is invariant with respect to taking pullbacks (resp. pushouts). The same is true for acyclic (co)fibrations.
\end{corollary}
\begin{proof}
	Let us only show the case with fibrations: let $p\colon A \fib B$ be a fibration, then the universal property of the pullback implies that if $p$ has the RLP with respect to a map, then we can say the same for the pushout (see the diagram below).
	\[\begin{tikzcd}
		X & {A\times_{B'}B} & A \\
		Y & B & B
		\arrow[from=1-1, to=1-2]
		\arrow["i", hook, from=1-1, to=2-1]
		\arrow[from=1-2, to=1-3]
		\arrow[from=1-2, to=2-2]
		\arrow["p", two heads, from=1-3, to=2-3]
		\arrow[dashed, from=2-1, to=1-2]
		\arrow[dashed, from=2-1, to=1-3]
		\arrow[from=2-1, to=2-2]
		\arrow[from=2-2, to=2-3]
	\end{tikzcd}\]
\end{proof}

\begin{definition}[Fibrant and cofibrant objects]
	An object $A$ is said to be \textbf{fibrant} if the terminal map $A \longrightarrow *$ is a fibration; an object $B$ is said to be \textbf{cofibrant} if the initial map $\emptyset \longrightarrow B$ is a cofibration.
\end{definition}

\begin{remark}
	Let $B$ be a cofibrant object, given that the coproduct $A \amalg A$ can be seen as the pushout with respect to the initial object, we deduce that both $\inn_1$ and $\inn_2$ must be cofibrations.
	\[\begin{tikzcd}
		\emptyset & B \\
		B & {B\amalg B}
		\arrow[from=1-1, to=1-2]
		\arrow[from=1-1, to=2-1]
		\arrow["{\inn_1}", from=1-2, to=2-2]
		\arrow["{\inn_2}"', from=2-1, to=2-2]
	\end{tikzcd}\]
	The dual result is also true: if $A$ is fibrant then both projections
	\[\pi_1,\pi_2 \colon A\times A \longrightarrow A\]
	must be fibrations.
\end{remark}


\begin{definition}[Abelian model categories]
	
\end{definition}


\section{Homotopy relations}

\begin{definition}[Cylinder objects]
	let $A$ be an object of $\C$; a \textbf{cylinder object} is and object $A \wedge I$ (when we consider multiple cylinders we can add superscripts/subscripts to $I$), and a diagram
	\[\begin{tikzcd}
		{A\amalg A} & {A\wedge I} & {A.}
		\arrow["i", from=1-1, to=1-2]
		\arrow["{\nabla_A}"', curve={height=24pt}, from=1-1, to=1-3]
		\arrow["q", "\sim"', from=1-2, to=1-3]
	\end{tikzcd}\]
	A cylinder object is a \textbf{good cylinder object} if the map
	\[\begin{tikzcd}
		{A\amalg A} & {A\wedge I}
		\arrow["i", from=1-1, to=1-2]
	\end{tikzcd}\]
	is a cofibration.\\
	A cylinder object is a \textbf{very good cylinder object} if the map
	\[\begin{tikzcd}
		{A\wedge I} & {A.}
		\arrow["q", "\sim"', from=1-1, to=1-2]
	\end{tikzcd}\]
	is an acyclic fibration. We also denote the maps $A \longrightarrow A \wedge I$ as follows:
	\[\begin{tikzcd}
		A \\
		& {A \amalg A} & {A\wedge I.} \\
		A
		\arrow["{\inn_1}"', from=1-1, to=2-2]
		\arrow["{i_1}"', curve={height=-18pt}, from=1-1, to=2-3]
		\arrow["i", from=2-2, to=2-3]
		\arrow["{\inn_2}", from=3-1, to=2-2]
		\arrow["{i_2}", curve={height=18pt}, from=3-1, to=2-3]
	\end{tikzcd}\]
\end{definition}

It is important to observe that for every object $A$ a very good cylinder object exists always: we only need to apply (\textbf{MC4}) to $\nabla_A \colon A \amalg A \longrightarrow A$. 

\begin{remark}
	Given a cylinder $A\wedge I$, we know that $q \circ i_1 = q \circ i \circ\inn_1 = \nabla_A \circ \inn_1 = \id_A$ and both $\id_A$ and $q$ are weak equivalences, therefore by (\textbf{MC1}) we assert that $i_1$ must also be a weak equivalence (and the same can be said for $i_2$). 
	\[\begin{tikzcd}
		A \\
		{A \amalg A} & {A \wedge I} & A \\
		A
		\arrow["{\inn_1}"', from=1-1, to=2-1]
		\arrow["{i_1}", from=1-1, to=2-2]
		\arrow["{\id_A}", curve={height=-12pt}, from=1-1, to=2-3]
		\arrow["i", from=2-1, to=2-2]
		\arrow["q", "\sim"', from=2-2, to=2-3]
		\arrow["{\inn_2}", from=3-1, to=2-1]
		\arrow["{i_2}"', from=3-1, to=2-2]
		\arrow["{\id_A}"', curve={height=12pt}, from=3-1, to=2-3]
	\end{tikzcd}\]
	If $A \wedge I$ is a good cylinder and $A$ is cofibrant then $i_1= i \circ \inn_1$ is the composition of cofibrations, therefore $i_1$ (and $i_2$) must be acyclic cofibrations.
\end{remark}

\begin{definition}[Homotopies]
	Two morphisms $f\colon A \longrightarrow B$ are said to be \textbf{homotopic} if there exist a cylinder $A\wedge I$ and a map $H\colon A\wedge I \longrightarrow B$ such that
	\[\begin{tikzcd}
		A \\
		& {A \wedge I} && B \\
		A
		\arrow["{i_1}"', from=1-1, to=2-2]
		\arrow["f", curve={height=-12pt}, from=1-1, to=2-4]
		\arrow["H", from=2-2, to=2-4]
		\arrow["{i_2}", from=3-1, to=2-2]
		\arrow["g"', curve={height=12pt}, from=3-1, to=2-4]
	\end{tikzcd}\]
	commutes. We will refer to the diagram as an \textbf{homotopy}. \\
	An homotopy is said to be \textbf{good} (resp. \textbf{very good}) if the cylinder is good (resp. very good).
\end{definition}



