\chapter{Cotorsion pairs and model categories}
The main objective in this chapter will be obtaining a method to define model structures on abelian categories, by identifying the objects that we want to be fibrant/cofibrant. The reason we do so is to have more control on (co)fibrant replacement.
An in-depth treatment of the subject can be found in \cite{bullones2016introduction}.

\section{Small object argument}
One of the most common tools used to build a model structure on a given category is known as \virgolette{Quillen's small object argument}; it provides a tool to obtain a (functorial) factoriziation as required in (\textbf{MC4}). We will not provide a proof, to see the full result an excellent resource is chapter 2 of \cite{hovey2007model}.\\
Let $\C$ be a category.

\begin{definition}[Injective and projective morphisms]
	Let $I$ be a class of morphisms of $\C$. 
	\begin{itemize}
		\item A morphism is said to be $I$\textbf{-injective} if it has the RLP with respect to every morphism in $I$. The class of $I$-injective morphisms is denoted $I$-inj.
		\item A morphism is said to be $I$\textbf{-projective} if it has the LLP with respect to every morphism in $I$. The class of $I$-projective morphisms is denoted $I$-proj.
		\item A morphism is said to be $I$\textbf{-cofibration} if it has the LLP with respect to every morphism in $I$-inj. The class of $I$-cofibrations is ($I$-inj)-proj and is denoted $I$-cof.
		\item A morphism is said to be $I$\textbf{-fibration} if it has the RLP with respect to every morphism in $I$-proj. The class of $I$-cofibrations is ($I$-proj)-inj and is denoted $I$-fib.
	\end{itemize}
\end{definition}

As an example, if $\C$ is a model category and $I = \cofT(\C)$, proposition \ref{prop:LLP-RLP} says that $I$-inj is exactly the class of acyclic fibrations of $\C$ and that $I$-cof is exactly $I$. Dually, if $I = \fibT(\C)$, $I$-proj is exactly the class of acyclic cofibrations of $\C$ and that $I$-fib is exactly $I$.

\begin{remark}
	In definition \ref{def:inj-proj}, we have defined injective and projective objects in an abelian category: if we let $I$ be the class of monomorphisms of $\C$, then an object $Y$ is injective if and only if the terminal morphism $Y \longrightarrow 0$ is in $I$-inj. Dually if we let $I$ be the class of epimorphisms of $\C$, then an object $X$ is projective if and only if the initial morphism $0 \longrightarrow X$ is in $I$-proj.
\end{remark}

It is easy to verify that $I \subseteq I\text{-cof}$ and $I \subseteq I\text{-fib}$; as a consequence we have that $(I\text{-cof})\text{-inj} = I \text{-inj}$; furthermore it is also clear that if $I \subseteq J$ we have that $J \text{-inj} 	\subseteq I \text{-inj}$ and $J \text{-proj} 	\subseteq I \text{-proj}$. From these two results follows that $(I\text{-cof})\text{-inj} = I \text{-inj}$ and $(I\text{-fib})\text{-proj} = I \text{-proj}$.  \\
Let us assume from now that $\C$ has all (small) colimits.

\begin{definition}[$\lambda$-sequences]
	Let $\lambda$ be an ordinal. A $\lambda$\textbf{-sequence} in $\C$ is a colimit preserving functor $X \colon \lambda \longrightarrow \C$ (where $\lambda$ is seen as a partial order); it is commonly written as
	\[\begin{tikzcd}
		{X_0} & {X_1} & \cdots & {X_\beta} & {\cdots.}
		\arrow[from=1-1, to=1-2]
		\arrow[from=1-2, to=1-3]
		\arrow[from=1-3, to=1-4]
		\arrow[from=1-4, to=1-5]
	\end{tikzcd}\]
	We refer to the colimit morphism $X_0 \longrightarrow \colim_{\beta < \lambda}(X_\beta)$ as the \textbf{transifinite compositions} of the morphisms $X_\beta \longrightarrow X_{\beta +1}$.
\end{definition}
Given that for every limit ordinal $\beta$ we know that $\colim_{\alpha < \beta} (\alpha) = \beta$, for every $\lambda$-sequence $X$ and for every limit ordinal $\beta < \lambda$ the canonical morphism
\[\colim_{\alpha < \beta} X \longrightarrow X_\beta,\]
is an isomorphism.


\begin{remark}
	One important result about (AB5) categories is the fact that monomorphism are closed under transifinite composition; it can be proven by transifinite induction and the only non trivial step is the one for limit ordinals: if $X$ is a $\lambda$-sequence made up of monomorphisms and $\beta$ is a limit ordinal, using the fact that ordinals are filtered and colimits are exact in (AB5) categories, we have that 
	\[\ker(X_0 \longrightarrow X_\beta) = \colim_{\alpha< \beta} \ker(X_0 \longrightarrow X_\alpha) =  \colim_{\alpha< \beta} 0 = 0.\]
\end{remark}


\begin{definition}[Filtered ordinal]
	Let $\gamma$ be a cardinal; an ordinal $\alpha$ is said to be $\gamma$\textbf{-filtered} if for every $A\subseteq \alpha$ such that $|A| \leq \gamma$ we have $\text{sup} \: A < \alpha$.
\end{definition}

\begin{definition}[Small objects]
	Let $I$ be a class of morphisms of $\C$, $A$ an object of $\C$ and $\kappa$ a cardinal. We say that $A$ is $\kappa$\textbf{-small relative} to $I$ if, for all $\kappa$-filtered ordinals $\lambda$ and all $\lambda$ sequences $X_\beta$ such that $X_\beta \longrightarrow X_{\beta+1}$ is in $I$ for $\beta < \lambda$, the canonical map 
	\[\colim_{\beta< \lambda} \Hom_\C(A, X_\beta) \longrightarrow \Hom_{\C}(A, \colim_{\beta < \lambda} X_\beta)\]
	is bijective. We say that $A$ is \textbf{small relative} $I$ if it is $\kappa$-small relative to $I$ for some $\kappa$. We say that $A$ is \textbf{small} if it is small relative to the class of morphisms of $\C$.
\end{definition}




To avoid dealing with set-theoretic issues, our seetting will often be that of Grothendieck categories, due to the proposition below.

\begin{proposition}
	Every object in a Grothendieck category is small.
\end{proposition}
\begin{proof}
	See proposition 1.2 of \cite{HoveySheaves1999}.
\end{proof}

\begin{definition}[Relative cell complex]
	Let $I$ be a set of morphisms. A \textbf{relative} $I$\textbf{-cell complex} is a transifinite composition of pushouts of elements of $I$. We denote the class of relative $I$-cell complexes as $I$-cell. We say that an object $A\in \C$ is a $I$\textbf{-cell complex} if the initial morphism $\emptyset \longrightarrow A$ is a relative $I$-cell complex.
\end{definition}

\begin{lemma}
	Let $I$ be a class of morphisms in $\C$. We have $I\text{-cell} \subseteq I\text{-cof}$.
\end{lemma}
\begin{proof}
	We already know that $I \subseteq I\text{-cof}$; given that $I$-cof is defined by a lifting property it is easy to show that $I$-cof is closed under finite compositions and pushouts, by proceeding as we did in corollary \ref{cor:closed-comp} and \ref{cor:closed-push-pull}. It reamains only to show that $I$-cof is closed under transifinite composition, to prove that it is as such we will argue by transifinite induction: let $\lambda$ be an ordinal and let $X_\beta$ be a $\lambda$ sequence such that $X_\beta \longrightarrow X_{\beta+1}$ is in $I$-cof. Let $\beta+1 \leq \lambda$ be a successor ordinal, if we assume that $X_0 \longrightarrow X_\beta$ is in $I$-cof then $X_0 \longrightarrow X_{\beta+1}$ must also be in $I$-cof (given that $I$-cof is closed under finite composition). Let now $\beta \leq \lambda$ be a limit ordinal and let's assume that for every $\alpha < \beta$ we have that $X_0 \longrightarrow X_\alpha$ is in $I$-cof; let $A\longrightarrow B$ be in $I$-inj, we want to show that $X_0 \longrightarrow X_\beta$ has the LLP with respect to it, but this follows from the universal property of the colimit.
	\[\begin{tikzcd}
		{X_0} & A \\
		{X_\alpha} \\
		{X_\beta \cong \colim_{\beta<\alpha}X_\alpha} & B
		\arrow[from=1-1, to=1-2]
		\arrow[from=1-1, to=2-1]
		\arrow[from=1-2, to=3-2]
		\arrow[dashed, from=2-1, to=1-2]
		\arrow[from=2-1, to=3-1]
		\arrow[dashed, from=3-1, to=1-2]
		\arrow[from=3-1, to=3-2]
	\end{tikzcd}\]
	
\end{proof}

The following theorem, due to Quillen, is one of the main tools used to obtain (functorial) factorizations of a morphism (i.e. a method to prove (\textbf{MC4})), which is usually the hardest part of defining a model structure on a category.

\begin{theorem}[The small object argument]
	Let $\C$ be a category with small colimits amd let $I$ be a set of morphisms of $\C$. If the domains of the elements of $I$ are small with respect to $I$-cell, then there exists a (functorial) factorization $(\gamma, \delta)$ such that, for every morphism $f$, $\gamma(f)$ is in $I$-cell and $\delta(f)$ is in $I$-inj.
\end{theorem}

Model categories constructed through the small object argument enjoy peculiar properties, we will collect them in the following definition.

\begin{definition}[Cofibrantly generated model categories]
	Let $\C$ be a model category. $\C$ is \textbf{cofibrantly generated} if there are sets $I,J$ of morphisms of $\C$ such that:
	\begin{enumerate}
		\item The domains of the morphisms in $I$ are small relative to $I$-cell.
		\item The domains of the morphisms in $J$ are small relative to $J$-cell.
		\item The class of fibrations of $\C$ is $J$-inj.
		\item The class of acyclic fibrations of $\C$ is $I$-inj. 
	\end{enumerate}
	We will refer to $I$ as the set of generating cofibrations and to $J$ as the set of generating acyclic cofibraions.
\end{definition}

The main properties of cofibrantly generated model categories are stated in the following proposition.

\begin{proposition}
	Let $\C$ be cofibrantly generated model category, with generating cofibrations $I$ and generating cofibrations $J$. The following holds:
	\begin{enumerate}
		\item The cofibrations are the class $I$-cof.
		\item Every cofibration is a retract of a relative $I$-cell complex.
		\item The domains of morphisms in $I$ are small relative with respect to the cofibrations.
		\item The trivial cofibrations are the class $J$-cof.
		\item Every  trivial cofibration is a retract of a relative $J$-cell complex.
		\item The domains of morphisms in $J$ are small relative with respect to the trivial cofibrations.
	\end{enumerate}
\end{proposition}

The theorem below gives us a method to build cofibrantly generated model categories through the small object argument.

\begin{theorem}
	Let $\C$ be a category with (small) limits and colimits, let $W$ be a class of morphism of $\C$ and let $I,J$ be sets of morphism of $\C$. Then there is a cofibrantly generated model structure on $\C$ with $I$ as the set of generating cofibrations, $J$ as the set of generating trivial cofibrations, and $W$ as the subcategory of weak equivalences if and only if the following conditions are satisfied:
	\begin{enumerate}
		\item The class $W$ satisfy the two-out-of-three property (\textbf{MC1}) and is closed under retracts.
		\item The domains of $I$ are small relative to $I$-cell.
		\item The domains of $J$ are small relative to $J$-cell.
		\item $J\text{-cell} \subseteq W \cap I\text{-cof}$.
		\item $I\text{-inj} \subseteq W \cap J\text{-inj}$.
		\item Either $W \cap I\text{-cof} \subseteq $ or $ W \cap J\text{-inj} \subseteq I\text{-inj} $.
	\end{enumerate}
\end{theorem}

\section{Cotorsion pairs}
Let $\C$ be an abelian category.

\begin{definition}[orthogonal complement]
	Let $\D$ be a class of objects of $\C$. We define the \textbf{left orthogonal complement} of $\D$ as the class
	\[ \lperp{\D} = \left\{ E \: : \:  \Ext^1(E,D) = 0, \text{ for every } D \in \D\right\}.\]
	Dually, we define the \textbf{right orthogonal complement} of $\D$ as the class
	\[ \rperp{\D} = \left\{ E \: : \:  \Ext^1(D,E) = 0, \text{ for every } D \in \D\right\}.\]
\end{definition}
The easiest example of orthogonal complements is the following: if $\D = \ob(\C)$, then $\lperp{\D}$ is the class of projective objects of $\C$, while $\rperp{\D}$ is the class of injective objects.

\begin{remark}
	Given a class of objects $\D$, it is clear that $\D\subseteq \lperp{(\rperp{\D})}$ and $\D\subseteq \rperp{(\lperp{\D})}$. Also if $\D\subseteq \D'$ we have $\lperp \D' \subseteq \lperp \D$ and $\lperp \D' \subseteq \lperp \D$, therefore $\rperp{(\lperp{(\rperp{\D})})} = \rperp{\D}$ and $\lperp{(\rperp{(\lperp{\D})})} = \lperp{\D}$.
\end{remark}


\begin{definition}[Cotorsion pair]
	Let $(\D$, $\E)$ be a pair of classes of objects of $\C$, they are said to be a \textbf{cotorsion pair} if $\D = \lperp{\E}$ and $\E = \rperp{\D}$.
\end{definition}


One common way to define a cotorsion pair is to cogenerate it from a set, i.e. to choose a set $\D'$ and let $\E = \rperp{\D'}$ and $\D = \lperp{\E}$; the above remark proves that it is actually a cotorsion pair. The dual notion, generating the cotorsion pair from a set $\E' \subseteq \E$, does not commonly arise in practice, due to the asimmetry of certain theorems that we will need later. \\


The example we used earlier allows us to construct two examples of cotorsion pairs: $(\proj(\C), \ob(\C))$ and $(\ob(\C),\inj(\C))$. Given these examples, we would like to generalize the concept of \virgolette{enough injectives} and \virgolette{enough projectives} to cotorsion pairs.



\begin{definition}[Complete cotorsion pair]
	A cotorsion pair $(\D, \E)$ is said to \textbf{have enough injectives} if for every $X \in \ob(\C)$ there exists an an exact sequence
	\[\begin{tikzcd}
		0 & X & E & D & {0,}
		\arrow[from=1-1, to=1-2]
		\arrow[from=1-2, to=1-3]
		\arrow[from=1-3, to=1-4]
		\arrow[from=1-4, to=1-5]
	\end{tikzcd}\]
	where $E\in \E$ and $D \in \D$. Dually, a cotorsion pair $(\D, \E)$ is said to \textbf{have enough projectives} if for every $X \in \ob(\C)$ there exists an an exact sequence
	\[\begin{tikzcd}
		0 & E & D & X & {0,}
		\arrow[from=1-1, to=1-2]
		\arrow[from=1-2, to=1-3]
		\arrow[from=1-3, to=1-4]
		\arrow[from=1-4, to=1-5]
	\end{tikzcd}\]
	where $E\in \E$ and $D \in \D$. A cotorsion pair that has both enough injectives and enough projectives is said to be \textbf{complete}.
\end{definition}



\begin{definition}
	Let $\D$ be a class of objects in $\C$, let $\lambda$ be an ordinal and let $X\colon \lambda \longrightarrow \C$ be a $\lambda$-sequence. If for any $\beta < \lambda$ the morphism $X_\beta \longrightarrow X_{\beta +1}$ is monic and its cokernel is in $\D$, we say that $X_0\longrightarrow \colim_{\beta < \lambda} X$ is a \textbf{transifinite extension} of $X_0$ by $\D$. If it is also true that $X_0 \in \D$, we simply refer to $\colim_{\beta < \lambda} X$ as a transifinite extension of $\D$.
\end{definition}

\begin{lemma}
	\label{lem:cotorsion-trans}
	Let us assume that $\C$ has all (small) limits and colimits; if $(\D, \E)$ is a cotorsion pair then $\D$ is closed under retracts and transifinite extensions.
\end{lemma}
\begin{proof}
	The statement about retract is obvious: given that $\Ext^1({-},Y)$ is an additive functor, if $X$ is a retract of $X'$ then $\Ext^1(X,Y)$ is a retract of $\Ext^1(X',Y)$.
	We will now prove that for any $Y \in \ob(\C)$, the class of objects $X$ such that $\Ext^1(X,Y)=0$ is closed under transifinite composition. Let $\lambda$ be an ordinal and let us consider a $\lambda$-sequence $X\colon \lambda \longrightarrow \C$ such that $\Ext^1(X_{\beta +1} / X_\beta,Y) = 0$ and $i_\alpha \colon X_\beta \longrightarrow X_ {\beta+1}$ is monic for any $\beta < \lambda$. We will now use transifinite induction to prove that $\Ext^1(X_\beta, Y) = 0$ for any $\beta \leq \lambda$, where we define $X_\lambda := \colim_{\beta < \lambda} X$. The base step $X_0$ is trivial. If $\beta := \alpha+1$ is a successor ordinal, we  have the exact sequence
	\[\begin{tikzcd}
		0 & {X_\alpha} & {X_{\alpha +1}} & {X_{\alpha+1}/ X_\alpha} & {0,}
		\arrow[from=1-1, to=1-2]
		\arrow[from=1-2, to=1-3]
		\arrow[from=1-3, to=1-4]
		\arrow[from=1-4, to=1-5]
	\end{tikzcd}\]
	that induces
	\[\begin{tikzcd}
		{0 \cong\Ext^1(X_{\alpha+1}/ X_\alpha, Y)} & {\Ext^1(X_{\alpha +1}, Y)} & {\Ext^1( X_\alpha, Y) \cong 0.}
		\arrow[from=1-1, to=1-2]
		\arrow[from=1-2, to=1-3]
	\end{tikzcd}\]
	The case where $\beta$ is a successor ordinal is a little more involved: we want to show that all exact sequences of the form
	\[\begin{tikzcd}
		0 & Y & A & {X_\beta} & 0
		\arrow[from=1-1, to=1-2]
		\arrow["f", from=1-2, to=1-3]
		\arrow["g", from=1-3, to=1-4]
		\arrow[from=1-4, to=1-5]
	\end{tikzcd}\]
	are splitting. By pulling back the extensions along the morphisms $X_\alpha \longrightarrow X_\beta $ and by setting $A_\alpha := A \times_{X_\beta} X_\alpha$, we get a family of compatible short exact sequences as follows.
	\[\begin{tikzcd}[sep=small]
		& 0 && Y && {A_\alpha} && {X_\alpha} && 0 \\
		0 && Y && {A_{\alpha+1}} && {X_{\alpha+1}} && 0 \\
		& 0 && Y && A && {X_\beta} && 0
		\arrow[from=1-2, to=1-4]
		\arrow["{{f_\alpha}}", from=1-4, to=1-6]
		\arrow["{\id_Y}"', from=1-4, to=2-3]
		\arrow["{\id_Y}"{pos=0.8}, from=1-4, to=3-4]
		\arrow["{{g_\alpha}}", from=1-6, to=1-8]
		\arrow["{{j_\alpha}}"', from=1-6, to=2-5]
		\arrow[from=1-6, to=3-6]
		\arrow[from=1-8, to=1-10]
		\arrow["{{i_\alpha}}"', from=1-8, to=2-7]
		\arrow[from=1-8, to=3-8]
		\arrow[from=2-1, to=2-3]
		\arrow["{{f_{\alpha+1}}}"{pos=0.75}, from=2-3, to=2-5]
		\arrow["{\id_Y}"', from=2-3, to=3-4]
		\arrow["{{g_{\alpha+1}}}"{pos=0.75}, from=2-5, to=2-7]
		\arrow[from=2-5, to=3-6]
		\arrow[from=2-7, to=2-9]
		\arrow[from=2-7, to=3-8]
		\arrow[from=3-2, to=3-4]
		\arrow["f", from=3-4, to=3-6]
		\arrow["g", from=3-6, to=3-8]
		\arrow[from=3-8, to=3-10]
	\end{tikzcd}\]
	We know by inductive hypothesis that all of these sequences (except possibly the last) must split; we will now use another transfinite induction to show that we can choose all of the sections $s_\alpha \colon X_{\alpha} \longrightarrow A_\alpha$ in a compatible way (i.e. $j_{\alpha} s_\alpha = i_\alpha s_{\alpha+1}$), obtaining at last a splitting of the desired exact sequence. Given that there is no compatibility condition for $\alpha+1 = 0$, we are free to choose any section for $s_0$, so the base step of the induction is trivial. If $\alpha$ is a limit ordinal, thee section $s_\alpha$ will simply be the following colimit morphism.
	\[\begin{tikzcd}
		{X_{\alpha'}} & {A_{\alpha'}} & A \\
		& {\colim_{\alpha'<\alpha}X \cong X_\alpha}
		\arrow["{s_{\alpha'}}", from=1-1, to=1-2]
		\arrow[from=1-1, to=2-2]
		\arrow[from=1-2, to=1-3]
		\arrow["{s_\alpha}", dashed, from=2-2, to=1-3]
	\end{tikzcd}\]
	We now only need to show the successor step: given a section $s_\alpha \colon X_\alpha \longrightarrow A_\alpha$ of $g_\alpha$, we want to find a section $s_{\alpha +1} \colon X_{\alpha+1} \longrightarrow A_{\alpha+1} $ such that $j_\alpha s_\alpha = s_{\alpha+1}i_{\alpha}$. Given that we already know that the sequence ending in $X_{\alpha+1}$ must split, there exists a section $t_{\alpha+1}$ of $g_{\alpha+1}$. Given that
	\[g_{\alpha+1}(j_\alpha s_\alpha - t_{\alpha+1} i_\alpha) = i_\alpha-i_\alpha = 0\]
	and that $Y$ is the kernel of $g_{\alpha+1}$, there is $h\colon X_\alpha\longrightarrow Y$ that satisfies $f_{\alpha+1} h = j_\alpha s_\alpha -t_{\alpha+1} i_{\alpha}$. The fact that $\Ext^1(X_{\alpha+1}/X_\alpha, Y) = 0$ implies that the map \[\Hom(X_{\alpha+1}, Y) \longrightarrow \Hom(X_{\alpha}, Y)\] is surjective, therefore there exists $k \colon X_{\alpha+1} \longrightarrow Y$ such that $ki_\alpha = h$. It is easy to check that $s_{\alpha+1} := t_{\alpha+1} +f_{\alpha+1}k$ is the section we are looking for.
\end{proof}






\section{Hovey correspondence}

We would like to use cotorsion pairs to define model structures on abelian categories, in a way that makes such structure compatible with the usual tools we use to study abelian categories. To do so we must fist make explicit these compatibility conditions.


\begin{definition}[Abelian model categories]
	Let $\C$ be a model category where the underlying category is abelian; we will refer to $\C$ as an \textbf{abelian model category} if the following hold:
	\begin{itemize}
		\item A morphism is a cofibration if and only if it is a monomorphism with cofibrant cokernel.
		\item A morphism is a fibration if and only if it is an epimorphism with fibrant kernel.
	\end{itemize}
\end{definition}

In an abelian model category we can give the following definition.

\begin{definition}[Acyclic object]
	An object $A$ is said to be \textbf{acyclic} if either zero morphism $A \longrightarrow 0$ or $0 \longrightarrow A$ is a weak equivalence. 
\end{definition}

\begin{lemma}
	\label{lem:ext-model}
	If $\C$ is an abelian model category, $C$ its cofibrant objects, $F$ its fibrant objects and $W$ its acyclic objects, then $\Ext^1(A,B) = 0$ when $A\in C$ and $B \in F\cap W$ or $A\in C\cap W$ and $B\in F$
\end{lemma}

\begin{proof}
	Let us consider $A \in C$ and $B \in F\cap W$, we want to show that $\Ext^1(A,B) = 0$. An element of $\Ext^1(A,B)$ is the equivalence class of a short exact sequence
	\[\begin{tikzcd}
		0 & B & X & A & {0.}
		\arrow[from=1-1, to=1-2]
		\arrow["f", from=1-2, to=1-3]
		\arrow["g", from=1-3, to=1-4]
		\arrow[from=1-4, to=1-5]
	\end{tikzcd}\]
	Given that $A$ is cofibrant and is the cokernel of $f$, $f$ is a cofibration. We can therefore find a lift in the following diagram,
	\[\begin{tikzcd}
		B & B \\
		X & {0}
		\arrow["{\id_B}", from=1-1, to=1-2]
		\arrow["f"', hook, from=1-1, to=2-1]
		\arrow["\sim", two heads, from=1-2, to=2-2]
		\arrow[dashed, from=2-1, to=1-2]
		\arrow[from=2-1, to=2-2]
	\end{tikzcd}\]
	which is a retraction of $f$; therefore the exact sequence must split and $\Ext^1(A,B) = 0$. The other case is analogous.
\end{proof}

The following proposition charcterizes acyclic fibrations/cofibrations in an abelian model category more explicitly.

\begin{proposition}
	\label{prop:acyc-in-ab}
	Let $\C$ be both an abelian category and an abelian category (not necessarily an abelian model category!) where every cofibration is a monomorphisms and every fibration is an epimorphism, the following two results hold: fibrations coincide with epimorphisms with fibrant kernels if and only if acyclic cofibrations coincide with monomorphisms with acyclic cofibrant cokernels; similarly, acyclic fibrations coincide with epimorphisms with acyclic fibrant kernels if and only if cofibrations
	coincide with monomorphisms with cofibrant cokernels.
\end{proposition}

\begin{proof}
	We will only consider the case where we assume that acyclic fibrations coincide with epimorphisms with acyclic fibrant kernels and we prove that cofibrations
	coincide with monomorphisms with cofibrant cokernels; the other three implications are analogous.
	We will proceed by considering a monomorphism $f\colon A \longrightarrow B$ and we will assume that $C = \coker(f)$ is cofibrant, therefore obtaining a ses as follows.
	\[\begin{tikzcd}
		0 & A & B & C & 0
		\arrow[from=1-1, to=1-2]
		\arrow["f", from=1-2, to=1-3]
		\arrow[from=1-3, to=1-4]
		\arrow[from=1-4, to=1-5]
	\end{tikzcd}\]
	We can now consider a commutative diagram
	\[\begin{tikzcd}
		A & X \\
		B & {Y,}
		\arrow["\alpha", from=1-1, to=1-2]
		\arrow["f", from=1-1, to=2-1]
		\arrow["g", "\sim"', two heads, from=1-2, to=2-2]
		\arrow["\beta", from=2-1, to=2-2]
	\end{tikzcd}\]
	where $g\colon X \longrightarrow Y$ is an acyclic fibration; our objective will be finding a lift in this diagram. If we set $Z = \ker(g)$ (and remember that it must be an acyclic fibrant object) we get another SES
	\[\begin{tikzcd}
		0 & Z & X & Y & {0.}
		\arrow[from=1-1, to=1-2]
		\arrow[from=1-2, to=1-3]
		\arrow["g", from=1-3, to=1-4]
		\arrow[from=1-4, to=1-5]
	\end{tikzcd}\]

	By carefully applying the functor $\Hom({-},{-})$ to the exact sequences above, we get the following commuting diagram with exact rows and columns.

	\[\begin{tikzcd}
		{\Hom(C,Z)} & {\Hom(C,X)} & {\Hom(C,Y)} & {\Ext^1(C,Z)} \\
		{\Hom(B,Z)} & {\Hom(B,X)} & {\Hom(B,Y)} & {\Ext^1(C,X)} \\
		{\Hom(A,Z)} & {\Hom(A,X)} & {\Hom(A,Y)} & {\Ext^1(A,Z)} \\
		{\Ext^1(C,Z)} & {\Ext^1(C,X)} & {\Ext^1(C,Y)} & \cdots
		\arrow[from=1-1, to=1-2]
		\arrow[from=1-1, to=2-1]
		\arrow["{g_*}", from=1-2, to=1-3]
		\arrow[from=1-2, to=2-2]
		\arrow["\delta", from=1-3, to=1-4]
		\arrow[from=1-3, to=2-3]
		\arrow[from=1-4, to=2-4]
		\arrow[from=2-1, to=2-2]
		\arrow["{f^*}", from=2-1, to=3-1]
		\arrow["{g_*}", from=2-2, to=2-3]
		\arrow["{f^*}", from=2-2, to=3-2]
		\arrow["\delta", from=2-3, to=2-4]
		\arrow["{f^*}", from=2-3, to=3-3]
		\arrow["{f^*}", from=2-4, to=3-4]
		\arrow[from=3-1, to=3-2]
		\arrow["\delta", from=3-1, to=4-1]
		\arrow["{g_*}", from=3-2, to=3-3]
		\arrow["\delta", from=3-2, to=4-2]
		\arrow["\delta", from=3-3, to=3-4]
		\arrow["\delta", from=3-3, to=4-3]
		\arrow[from=3-4, to=4-4]
		\arrow[from=4-1, to=4-2]
		\arrow["{g_*}", from=4-2, to=4-3]
		\arrow[from=4-3, to=4-4]
	\end{tikzcd}\]
	Our current objective is finding $\gamma\in \Hom(B,X)$ such that $f^* \gamma = \alpha$ and $g_* \gamma = \beta$, we will do so through some diagram chasing. We begin by noticing that \[g_*\delta \alpha = \delta g_* \alpha = \delta f^* \beta = 0. \] By lemma \ref{lem:ext-model} we know that ${\Ext^1(C,Z)} = 0$, therefore $g_* \colon{\Ext^1(C,X)} \longrightarrow {\Ext^1(C,Y)}$ is injective, thus $\delta \alpha = 0$ and by the exactness of the column we know that there exists $\gamma'\in \Hom(B,X)$ such that $f^* \gamma' = \alpha$. Given that $f^*(\beta - g_*\gamma')= f^*\beta -g_* \alpha = 0$, the morphism $\beta - g_*\gamma'$ must have a preimage $F$ in ${\Hom(C,Y)}$; using again lemma \ref{lem:ext-model}, we know that $g_*\colon {\Hom(C,X)} \longrightarrow  {\Hom(C,Y)}$ is surjective, and as a result there must be a morphism $G\in {\Hom(C,X)}$ such that $g_* G = F$. If we denote as $G'$ the image of $G$ in $\Hom(B,X)$, it is easy to check that $\gamma = \gamma' - G'$ is the morphism we are looking for.

\end{proof}



The objective of this section will be to prove \virgolette{Hovey’s Correspondence theeorem} (see section 2 of \cite{Hovey2002}), that shows a correspondence between abelian model structures and certain couples of cotorsion pairs. We will fix an abelian category $\C$ 

\begin{definition}[Thick class]
	A class of objects $W$ of $\C$ is said to be \textbf{thick} if it closed under retracts and whenever two out of three objects in a SES are in $W$ the third also is. 
\end{definition}

\begin{definition}[Hovey triple]
	Let $(C,W,F)$ be a triple of classes of objects in $\C$. We say that $(C,W,F)$ is an \textbf{Hovey triple} if $W$ is thick and $(C\cap W, F)$ and $(C, F\cap W)$ are complete cotorsion pairs.
\end{definition}



\begin{theorem}[Hovey's Correspondence - Part 1]
	If $\C$ is an abelian model category, $C$ its cofibrant objects, $F$ its fibrant objects and $W$ its acyclic objects, then $(C,W,F)$ is an Hovey triple.
\end{theorem}

\begin{proof}
	We begin by showing that $(C, F\cap W)$ is a complete cotorsion pair, given that proving it for $(C\cap W, F)$ is analogous. From lemma \ref{lem:ext-model}, it is immediate that $F\cap W \subseteq \rperp{C}$ and $C \subseteq \lperp{(F\cap W)}$, therefore to conclude the proof that $(C, F\cap W)$ is a cotorsion pair, we will also need to show that $\rperp{C} \subseteq  F\cap W$ and $\lperp{(F\cap W)} \subseteq C$; we will only prove the first of these two results (the other is dual). Let us now consider an object $B$ such that  $\Ext^1(A,B) = 0$ for every $A\in C$; then if we take any cofibration $f\colon X \longrightarrow Y$ we get an exact sequence
	\[\begin{tikzcd}
		0 & X & Y & {\coker(f)} & {0,}
		\arrow[from=1-1, to=1-2]
		\arrow["f", from=1-2, to=1-3]
		\arrow[from=1-3, to=1-4]
		\arrow[from=1-4, to=1-5]
	\end{tikzcd}\]
	where $\coker(f)$ must be a cofibrant object. Using the long exact sequence associated to the functor $\Hom({-},B)$ and the SES above, we get
	\[\begin{tikzcd}
		\cdots & {\Hom(X,B)} & {\Hom(Y,B)} & {\Ext^1(\coker(f),B)} & {\cdots;}
		\arrow[from=1-1, to=1-2]
		\arrow["{f^*}", from=1-2, to=1-3]
		\arrow[from=1-3, to=1-4]
		\arrow[from=1-4, to=1-5]
	\end{tikzcd}\]
	given that $\Ext^1(\coker(f),B)$ must be $0$, $f^*$ is surjective and therefore, by lemma \ref{lem:fib-to-surj}, $B$ must be in $F\cap W$. The last step we need is showing that this cotorsion pair is actually complete, but this is simply a consequence of (\textbf{MC4}): to show that $(C, F\cap W)$ has enough projectives, we can factor any initial morphism $0\longrightarrow X$ using an acyclic fibration $B \longrightarrow X$, where $B$ is cofibrant; taking the kernel of this morphism defines the exact sequence we need. Since the proof that shows that our cotorsion pair has enough injectives is identical, it must be complete.\\
	The last step that we need to complete this proof, is verifying that $W$ is a thick class. We know from axiom (\textbf{MC2}) that $W$ is closed under retracts, so we will only check the condition on SESes. 
	We begin by considering a SES as follows,
	\[\begin{tikzcd}
		0 & A & B & C & 0
		\arrow[from=1-1, to=1-2]
		\arrow["f", from=1-2, to=1-3]
		\arrow["g", from=1-3, to=1-4]
		\arrow[from=1-4, to=1-5]
	\end{tikzcd}\]
	then we can factor $g$ as 
	\[\begin{tikzcd}
		B & {B'} & C
		\arrow["j", "\sim"', hook, from=1-1, to=1-2]
		\arrow["{g'}", two heads, from=1-2, to=1-3]
	\end{tikzcd}\]
	to get a commuting diagram with exact rows as follows
	\[\begin{tikzcd}
		0 & A & B & C & 0 \\
		0 & {A'} & {B'} & C & {0,}
		\arrow[from=1-1, to=1-2]
		\arrow["f", from=1-2, to=1-3]
		\arrow["i", from=1-2, to=2-2]
		\arrow["g", from=1-3, to=1-4]
		\arrow["j", hook, from=1-3, to=2-3]
		\arrow[from=1-4, to=1-5]
		\arrow["{{\id_C}}", from=1-4, to=2-4]
		\arrow[from=2-1, to=2-2]
		\arrow["{{f'}}", from=2-2, to=2-3]
		\arrow["{{g'}}", two heads, from=2-3, to=2-4]
		\arrow[from=2-4, to=2-5]
	\end{tikzcd}\]
	where $A' = \coker(g')$ and $i$ is the morphism induced by the cokernel functor. By the snake lemma $\coker(i) = \coker{j}$ and so it is cofibrant, therefore $i$ is also a acyclic cofibration by proposition \ref{prop:acyc-in-ab}; we can also observe that $A'$ must be fibrant, given that it is the kernel of a fibration. If we now assume that $A \in W$ it is immediate that $A' \in W$, thus $g'$ must be a trivial fibration; by (\textbf{MC1}) $g$ must be a weak equivalence, therefore $B\in W$ if and only if $C\in W$. The remaining case ($B,C \in W$ implies $A\in W$) is simply the same argument in reverse. 
\end{proof}








\begin{theorem}[Hovey's Correspondence - Part 2]
	If $\C$ is an abelian model category and $(C,W,F)$ is an Hovey triple, then there exists a unique abelian model structure on $\C$ where $C$ are the cofibrant objects, $F$ are the fibrant objects and $W$ are the acyclic objects.
\end{theorem}





\section{Small cotorsion pairs}

The main obstacle in proving that certain classes form a Hovey triple, which we need if we want to use Hovey's correspondence theorem, is proving the completeness of the cotorsion pairs. This section will show that, if we are operating under certain hypothesis, this condition is free; in these particular cases we will also see that the model structure induced by the Hovey triple is cofibrantly generated.

\begin{definition}[Small cotorsion pair]
	Let $\C$ be an abelian category that has all (small) limits and colimits, let $(\D, \E)$ be a cotorsion pair of $\C$. We say that $(\D, \E)$ is \textbf{small} if all the following properties are satisfied.
	\begin{enumerate}
		\itemsep0em 
		\item $\D$ contains a family of generators $\G$.
		\item $(\D, \E)$ is cogenerated by a set  $\U$.
		\item for each $U \in \U$, there is a monomorphism $i_U\colon B_U \longrightarrow A_U$ with cokernel $U$ such that, if $i_U^*\colon \Hom(B_U, X) \longrightarrow \Hom(A_U, X)$ is surjective for all $U\in \U$, then $X \in \E$.
	\end{enumerate}
	We refer to the set $I$ containing all of the morphisms $i_U$ for $U \in \U$ and all of the initial morphisms $0 \longrightarrow G$ for $G \in \G$ as the \textbf{generating monomorphisms} of the cotorsion pair.
\end{definition}

It is important to remark that if $(\D, \E)$ is a small cotorsion pair, the converse of condition $3$ always holds: if $X \in \E$, then for all $U\in \U$ we have $\Ext^1(U, X) = 0$ and therefore by the long exact sequence associated to $\Ext$ we have that $i_U^*$ is surjective. We can use this fact to recover from the set $I$ of generating monomorphism the entire cotorsion pair: $\E$ is exactly the class of all objects $X$ such that $i^* = \Hom(i, X)$ is surjective for all $i\in I$ (the condition is trivial for morphisms of the form $0 \longrightarrow G$), while $\D = \lperp{\E}$. \\
From now on we will let $\C$ be a Grothendieck category.

\begin{theorem}
	Let $I$ be a set of monomorphisms of $\C$, then $I$ is the set of generating monomorphisms of a small cotorsion pair $(\D, \E)$ if and only if the following conditions hold:
	\begin{enumerate}
		\itemsep0em 
		\item $I$ contains the morphisms $0 \longrightarrow G_i$ for a generating set $\G = \{G_i\}$ of $\C$.
		\item For every $E \in \ob(\C)$ such that $i^*= \Hom(i, E)$ is surjective for all $i\in I$ and for every $j \in I$, we have $\Ext^1(\coker(j), E) = 0$  
	\end{enumerate}
	In this case, $\E$ is the class of all objects $E$ such that $i^*$ (as above) is surjective for all $i \in I$, while $\D$ is the smallest class closed under summands and transfinite extensions that contains all the cokernels of morphism in $I$. Furthermore $(\D, \E)$ is complete.
\end{theorem}

\begin{proof}
	If $I$ is a set of generating monomorphisms both conditions are easily verified. We will now prove that, given a set $I$ that satisfies both condition, the pair $(\D, \E)$ as in the statement of the theorem is a complete cotorsion pair.\\
	We begin by showing that this pair has enough injectives: by applying the small object argument to $X \longrightarrow 0$ we get a factorization of the form
	$\begin{tikzcd}[sep=small]
		X & Y & {0,}
		\arrow["f", from=1-1, to=1-2]
		\arrow["g", from=1-2, to=1-3]
	\end{tikzcd}$
	where $f$ is in $I$-cell and $g$ is in $I$-inj. This last result implies that $\Hom(i,Y)$ is surjective for all $i \in I$, therefore $Y \in \E$. Since monomorphisms are closed under pushouts and transifinite composition (we are in a Grothendieck category, which is AB3), we only need to show that $\coker(f)$ is in $\D$; it follows from the fact that $\coker(f)$ is a transfinite extension of cokernels of morphisms in $I$ (pushouts preserve cokernels) and the definition of $\D$. \\
	We will continue by showing that $(\D, \E)$ has enough injectives: this time we factor $0\longrightarrow X$ as 
	$\begin{tikzcd}[sep=small]
		0 & Z & {X,}
		\arrow["h", from=1-1, to=1-2]
		\arrow["k", from=1-2, to=1-3]
	\end{tikzcd}$
	where $h$ is in $I$-cell and $k$ is in $I$-inj. By proceeding as we did above, we get that $\coker(h) \cong Z$ must be in $\D$. Given that $0\longrightarrow G_i$ for $G_i \in G$ are in $I$ and $k$ is in $I$-inj, $k$ must be epic. The morphism $\ker(k) \longrightarrow 0$ is a pullback of of a morphism ($k$) in $I$-inj, therefore it is also in $I$-inj (see the proof of corollary \ref{cor:closed-push-pull}) and therefore, as we did earlier, $\ker(f)$ is in $\E$. \\
	The last step is proving that $(\D, \E)$ is actually a cotorsion pair.
\end{proof}

