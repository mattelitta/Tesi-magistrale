\chapter{Cotorsion pairs and model categories}
The main objective in this chapter will be obtaining a method to define model structures on abelian categories, by identifying the objects that we want to be fibrant/cofibrant. The reason we do so is to have more control on (co)fibrant replacement.
An in-depth treatment of the subject can be found in \cite{bullones2016introduction}.

\section{Small object argument}
One of the most common tools used to build a model structure on a given category is known as \virgolette{Quillen's small object argument}; it provides a tool to obtain a (functorial) factoriziation as required in (\textbf{MC4}). We will not provide a proof, to see the full result an excellent resource is chapter 2 of \cite{hovey2007model}.\\
Let $\C$ be a category.

\begin{definition}[Injective and projective morphisms]
	Let $I$ be a class of morphisms of $\C$. 
	\begin{itemize}
		\item A morphism is said to be $I$\textbf{-injective} if it has the RLP with respect to every morphism in $I$. The class of $I$-injective morphisms is denoted $I$-inj.
		\item A morphism is said to be $I$\textbf{-projective} if it has the LLP with respect to every morphism in $I$. The class of $I$-projective morphisms is denoted $I$-proj.
		\item A morphism is said to be $I$\textbf{-cofibration} if it has the LLP with respect to every morphism in $I$-inj. The class of $I$-cofibrations is ($I$-inj)-proj and is denoted $I$-cof.
		\item A morphism is said to be $I$\textbf{-fibration} if it has the RLP with respect to every morphism in $I$-proj. The class of $I$-cofibrations is ($I$-proj)-inj and is denoted $I$-fib.
	\end{itemize}
\end{definition}

As an example, if $\C$ is a model category and $I = \cofT(\C)$, proposition \ref{prop:LLP-RLP} says that $I$-inj is exactly the class of acyclic fibrations of $\C$ and that $I$-cof is exactly $I$. Dually, if $I = \fibT(\C)$, $I$-proj is exactly the class of acyclic cofibrations of $\C$ and that $I$-fib is exactly $I$.

\begin{remark}
	In definition \ref{def:inj-proj}, we have defined injective and projective objects in an abelian category: if we let $I$ be the class of monomorphisms of $\C$, then an object $Y$ is injective if and only if the terminal morphism $Y \longrightarrow 0$ is in $I$-inj. Dually if we let $I$ be the class of epimorphisms of $\C$, then an object $X$ is projective if and only if the initial morphism $0 \longrightarrow X$ is in $I$-proj.
\end{remark}

It is easy to verify that $I \subseteq I\text{-cof}$ and $I \subseteq I\text{-fib}$; as a consequence we have that $(I\text{-cof})\text{-inj} = I \text{-inj}$; furthermore it is also clear that if $I \subseteq J$ we have that $J \text{-inj} 	\subseteq I \text{-inj}$ and $J \text{-proj} 	\subseteq I \text{-proj}$. From these two results follows that $(I\text{-cof})\text{-inj} = I \text{-inj}$ and $(I\text{-fib})\text{-proj} = I \text{-proj}$.  \\
Let us assume from now that $\C$ has all (small) colimits.

\begin{definition}[$\lambda$-sequences]
	Let $\lambda$ be an ordinal. A $\lambda$\textbf{-sequence} in $\C$ is a colimit preserving functor $X \colon \lambda \longrightarrow \C$ (where $\lambda$ is seen as a partial order); it is commonly written as
	\[\begin{tikzcd}
		{X_0} & {X_1} & \cdots & {X_\beta} & {\cdots.}
		\arrow[from=1-1, to=1-2]
		\arrow[from=1-2, to=1-3]
		\arrow[from=1-3, to=1-4]
		\arrow[from=1-4, to=1-5]
	\end{tikzcd}\]
	We refer to the colimit morphism $X_0 \longrightarrow \colim_{\beta < \lambda}(X_\beta)$ as the \textbf{transifinite compositions} of the morphisms $X_\beta \longrightarrow X_{\beta +1}$.
\end{definition}
Given that for every limit ordinal $\beta$ we know that $\colim_{\alpha < \beta} (\alpha) = \beta$, for every $\lambda$-sequence $X$ and for every limit ordinal $\beta < \lambda$ the canonical morphism
\[\colim_{\alpha < \beta} X \longrightarrow X_\beta,\]
is an isomorphism.

\begin{definition}[Filtered ordinal]
	Let $\gamma$ be a cardinal; an ordinal $\alpha$ is said to be $\gamma$\textbf{-filtered} if for every $A\subseteq \alpha$ such that $|A| \leq \gamma$ we have $\text{sup} \: A < \alpha$.
\end{definition}

\begin{definition}[Small objects]
	Let $I$ be a class of morphisms of $\C$, $A$ an object of $\C$ and $\kappa$ a cardinal. We say that $A$ is $\kappa$\textbf{-small relative} to $I$ if, for all $\kappa$-filtered ordinals $\lambda$ and all $\lambda$ sequences $X_\beta$ such that $X_\beta \longrightarrow X_{\beta+1}$ is in $I$ for $\beta < \lambda$, the canonical map 
	\[\colim_{\beta< \lambda} \Hom_\C(A, X_\beta) \longrightarrow \Hom_{\C}(A, \colim_{\beta < \lambda} X_\beta)\]
	is bijective. We say that $A$ is \textbf{small relative} $I$ if it is $\kappa$-small relative to $I$ for some $\kappa$. We say that $A$ is \textbf{small} if it is small relative to the class of morphisms of $\C$.
\end{definition}

\begin{definition}[Relative cell complex]
	Let $I$ be a set of morphisms. A \textbf{relative} $I$\textbf{-cell complex} is a transifinite composition of pushouts of elements of $I$. We denote the class of relative $I$-cell complexes as $I$-cell. We say that an object $A\in \C$ is a $I$\textbf{-cell complex} if the initial morphism $\emptyset \longrightarrow A$ is a relative $I$-cell complex.
\end{definition}

\begin{lemma}
	Let $I$ be a class of morphisms in $\C$. We have $I\text{-cell} \subseteq I\text{-cof}$.
\end{lemma}
\begin{proof}
	We already know that $I \subseteq I\text{-cof}$; given that $I$-cof is defined by a lifting property it is easy to show that $I$-cof is closed under finite compositions and pushouts, by proceeding as we did in corollary \ref{cor:closed-comp} and \ref{cor:closed-push-pull}. It reamains only to show that $I$-cof is closed under transifinite composition, to prove that it is as such we will argue by transifinite induction: let $\lambda$ be an ordinal and let $X_\beta$ be a $\lambda$ sequence such that $X_\beta \longrightarrow X_{\beta+1}$ is in $I$-cof. Let $\beta+1 \leq \lambda$ be a successor ordinal, if we assume that $X_0 \longrightarrow X_\beta$ is in $I$-cof then $X_0 \longrightarrow X_{\beta+1}$ must also be in $I$-cof (given that $I$-cof is closed under finite composition). Let now $\beta \leq \lambda$ be a limit ordinal and let's assume that for every $\alpha < \beta$ we have that $X_0 \longrightarrow X_\alpha$ is in $I$-cof; let $A\longrightarrow B$ be in $I$-inj, we want to show that $X_0 \longrightarrow X_\beta$ has the LLP with respect to it, but this follows from the universal property of the colimit.
	\[\begin{tikzcd}
		{X_0} & A \\
		{X_\alpha} \\
		{X_\beta \cong \colim_{\beta<\alpha}X_\alpha} & B
		\arrow[from=1-1, to=1-2]
		\arrow[from=1-1, to=2-1]
		\arrow[from=1-2, to=3-2]
		\arrow[dashed, from=2-1, to=1-2]
		\arrow[from=2-1, to=3-1]
		\arrow[dashed, from=3-1, to=1-2]
		\arrow[from=3-1, to=3-2]
	\end{tikzcd}\]
	
\end{proof}

The following theorem, due to Quillen, is one of the main tools used to obtain (functorial) factorizations of a morphism (i.e. a method to prove (\textbf{MC4})), which is usually the hardest part of defining a model structure on a category.

\begin{theorem}[The small object argument]
	Let $\C$ be a category with small colimits amd let $I$ be a set of morphisms of $\C$. If the domains of the elements of $I$ are small with respect to $I$-cell, then there exists a (functorial) factorization $(\gamma, \delta)$ such that, for every morphism $f$, $\gamma(f)$ is in $I$-cell and $\delta(f)$ is in $I$-inj.
\end{theorem}

Model categories constructed through the small object argument enjoy peculiar properties, we will collect them in the following definition.

\begin{definition}[Cofibrantly generated model categories]
	Let $\C$ be a model category. $\C$ is \textbf{cofibrantly generated} if there are sets $I,J$ of morphisms of $\C$ such that:
	\begin{enumerate}
		\item The domains of the morphisms in $I$ are small relative to $I$-cell.
		\item The domains of the morphisms in $J$ are small relative to $J$-cell.
		\item The class of fibrations of $\C$ is $J$-inj.
		\item The class of acyclic fibrations of $\C$ is $I$-inj. 
	\end{enumerate}
	We will refer to $I$ as the set of generating cofibrations and to $J$ as the set of generating acyclic cofibraions.
\end{definition}

The main properties of cofibrantly generated model categories are stated in the following proposition.

\begin{proposition}
	Let $\C$ be cofibrantly generated model category, with generating cofibrations $I$ and generating cofibrations $J$. The following holds:
	\begin{enumerate}
		\item The cofibrations are the class $I$-cof.
		\item Every cofibration is a retract of a relative $I$-cell complex.
		\item The domains of morphisms in $I$ are small relative with respect to the cofibrations.
		\item The trivial cofibrations are the class $J$-cof.
		\item Every  trivial cofibration is a retract of a relative $J$-cell complex.
		\item The domains of morphisms in $J$ are small relative with respect to the trivial cofibrations.
	\end{enumerate}
\end{proposition}

The theorem below gives us a method to build cofibrantly generated model categories through the small object argument.

\begin{theorem}
	Let $\C$ be a category with (small) limits and colimits, let $W$ be a class of maps of $\C$ and let $I,J$ be sets of maps of $\C$. Then there is a cofibrantly generated model structure on $\C$ with $I$ as the set of generating cofibrations, $J$ as the set of generating trivial cofibrations, and $W$ as the subcategory of weak equivalences if and only if the following conditions are satisfied:
	\begin{enumerate}
		\item The class $W$ satisfy the two-out-of-three property (\textbf{MC1}) and is closed under retracts.
		\item The domains of $I$ are small relative to $I$-cell.
		\item The domains of $J$ are small relative to $J$-cell.
		\item $J\text{-cell} \subseteq W \cap I\text{-cof}$.
		\item $I\text{-inj} \subseteq W \cap J\text{-inj}$.
		\item Either $W \cap I\text{-cof} \subseteq $ or $ W \cap J\text{-inj} \subseteq I\text{-inj} $.
	\end{enumerate}
\end{theorem}

\section{Cotorsion pairs}
Let $\C$ be an abelian category.

\begin{definition}[orthogonal complement]
	Let $\D$ be a class of objects of $\C$. We define the \textbf{left orthogonal complement} of $\D$ as the class
	\[ \lperp{\D} = \left\{ E \: : \:  \Ext^1(E,D) = 0, \text{ for every } D \in \D\right\}.\]
	Dually, we define the \textbf{right orthogonal complement} of $\D$ as the class
	\[ \rperp{\D} = \left\{ E \: : \:  \Ext^1(D,E) = 0, \text{ for every } D \in \D\right\}.\]
\end{definition}
The easiest example of orthogonal complements is the following: if $\D = \ob(\C)$, then $\lperp{\D}$ is the class of projective objects of $\C$, while $\rperp{\D}$ is the class of injective objects.

\begin{remark}
	Given a class of objects $\D$, it is clear that $\D\subseteq \lperp{(\rperp{\D})}$ and $\D\subseteq \rperp{(\lperp{\D})}$. Also if $\D\subseteq \D'$ we have $\lperp \D' \subseteq \lperp \D$ and $\lperp \D' \subseteq \lperp \D$, therefore $\rperp{(\lperp{(\rperp{\D})})} = \rperp{\D}$ and $\lperp{(\rperp{(\lperp{\D})})} = \lperp{\D}$.
\end{remark}

One common way to define a cotorsion pair is to cogenerate it from a set, i.e. to choose a set $\D'$ and let $\E = \rperp{\D'}$ and $\D = \lperp{\E}$; the above remark proves that it is actually a cotorsion pair. The dual notion, generating the cotorsion pair from a set $\E' \subseteq \E$, does not commonly arise in practice, due to the asimmetryu of certain theorems that we will state later.

\begin{definition}[Cotorsion pair]
	Let $(\D$, $\E)$ be a pair of classes of objects of $\C$, they are said to be a \textbf{cotorsion pair} if $\D = \lperp{\E}$ and $\E = \rperp{\D}$.
\end{definition}

The example above allows us to construct two examples of cotorsion pairs: $(\proj(\C), \ob(\C))$ and $(\ob(\C),\inj(\C))$. Given these examples, we would like to generalize the concept of \virgolette{enough injectives} and \virgolette{enough projectives} to cotorsion pairs.



\begin{definition}[Complete cotorsion pair]
	A cotorsion pair $(\D, \E)$ is said to \textbf{have enough injectives} if for every $X \in \ob(\C)$ there exists an an exact sequence
	\[\begin{tikzcd}
		0 & X & E & D & {0,}
		\arrow[from=1-1, to=1-2]
		\arrow[from=1-2, to=1-3]
		\arrow[from=1-3, to=1-4]
		\arrow[from=1-4, to=1-5]
	\end{tikzcd}\]
	where $E\in \E$ and $D \in \D$. Dually, a cotorsion pair $(\D, \E)$ is said to \textbf{have enough projectives} if for every $X \in \ob(\C)$ there exists an an exact sequence
	\[\begin{tikzcd}
		0 & E & D & X & {0,}
		\arrow[from=1-1, to=1-2]
		\arrow[from=1-2, to=1-3]
		\arrow[from=1-3, to=1-4]
		\arrow[from=1-4, to=1-5]
	\end{tikzcd}\]
	where $E\in \E$ and $D \in \D$. A cotorsion pair that has both enough injectives and enough projectives is said to be \textbf{complete}.
\end{definition}




\section{Hovey correspondence}

We would like to use cotorsion pairs to define model structures on abelian categories, in a way that makes such structure compatible with the usual tools we use to study abelian categories. To do so we must fist make explicit these compatibility conditions.


\begin{definition}[Abelian model categories]
	Let $\C$ be a model category where the underlying category is abelian; we will refer to $\C$ as an \textbf{abelian model category} if the following hold:
	\begin{itemize}
		\item A morphism is a cofibration if and only if it is a monomorphism with cofibrant cokernel.
		\item A morphism is a fibration if and only if it is an epimorphism with fibrant kernel.
	\end{itemize}
\end{definition}

In an abelian model category we can give the following definition.

\begin{definition}[Acyclic object]
	An object $A$ is said to be \textbf{acyclic} if either zero map $A \longrightarrow 0$ or $0 \longrightarrow A$ is a weak equivalence. 
\end{definition}

\begin{lemma}
	\label{lem:ext-model}
	If $\C$ is an abelian model category, $C$ its cofibrant objects, $F$ its fibrant objects and $W$ its acyclic objects, then $\Ext^1(A,B) = 0$ when $A\in C$ and $B \in F\cap W$ or $A\in C\cap W$ and $B\in F$
\end{lemma}

\begin{proof}
	Let us consider $A \in C$ and $B \in F\cap W$, we want to show that $\Ext^1(A,B) = 0$. An element of $\Ext^1(A,B)$ is the equivalence class of a short exact sequence
	\[\begin{tikzcd}
		0 & B & X & A & {0.}
		\arrow[from=1-1, to=1-2]
		\arrow["f", from=1-2, to=1-3]
		\arrow["g", from=1-3, to=1-4]
		\arrow[from=1-4, to=1-5]
	\end{tikzcd}\]
	Given that $A$ is cofibrant and is the cokernel of $f$, $f$ is a cofibration. We can therefore find a lift in the following diagram,
	\[\begin{tikzcd}
		B & B \\
		X & {0}
		\arrow["{\id_B}", from=1-1, to=1-2]
		\arrow["f"', hook, from=1-1, to=2-1]
		\arrow["\sim", two heads, from=1-2, to=2-2]
		\arrow[dashed, from=2-1, to=1-2]
		\arrow[from=2-1, to=2-2]
	\end{tikzcd}\]
	which is a retraction of $f$; therefore the exact sequence must split and $\Ext^1(A,B) = 0$. The other case is analogous.
\end{proof}

The following proposition charcterizes acyclic fibrations/cofibrations in an abelian model category more explicitly.

\begin{proposition}
	\label{prop:acyc-in-ab}
	Let $\C$ be both an abelian category and an abelian category (not necessarily an abelian model category!) where every cofibration is a monomorphisms and every fibration is an epimorphism, the following two results hold: fibrations coincide with epimorphisms with fibrant kernels if and only if acyclic cofibrations coincide with monomorphisms with acyclic cofibrant cokernels; similarly, acyclic fibrations coincide with epimorphisms with acyclic fibrant kernels if and only if cofibrations
	coincide with monomorphisms with cofibrant cokernels.
\end{proposition}

\begin{proof}
	We will only consider the case where we assume that acyclic fibrations coincide with epimorphisms with acyclic fibrant kernels and we prove that cofibrations
	coincide with monomorphisms with cofibrant cokernels; the other three implications are analogous.
	We will proceed by considering a monomorphism $f\colon A \longrightarrow B$ and we will assume that $C = \coker(f)$ is cofibrant, therefore obtaining a ses as follows.
	\[\begin{tikzcd}
		0 & A & B & C & 0
		\arrow[from=1-1, to=1-2]
		\arrow["f", from=1-2, to=1-3]
		\arrow[from=1-3, to=1-4]
		\arrow[from=1-4, to=1-5]
	\end{tikzcd}\]
	We can now consider a commutative diagram
	\[\begin{tikzcd}
		A & X \\
		B & {Y,}
		\arrow["\alpha", from=1-1, to=1-2]
		\arrow["f", from=1-1, to=2-1]
		\arrow["g", "\sim"', two heads, from=1-2, to=2-2]
		\arrow["\beta", from=2-1, to=2-2]
	\end{tikzcd}\]
	where $g\colon X \longrightarrow Y$ is an acyclic fibration; our objective will be finding a lift in this diagram. If we set $Z = \ker(g)$ (and remember that it must be an acyclic fibrant object) we get another SES
	\[\begin{tikzcd}
		0 & Z & X & Y & {0.}
		\arrow[from=1-1, to=1-2]
		\arrow[from=1-2, to=1-3]
		\arrow["g", from=1-3, to=1-4]
		\arrow[from=1-4, to=1-5]
	\end{tikzcd}\]

	By carefully applying the functor $\Hom({-},{-})$ to the exact sequences above, we get the following commuting diagram with exact rows and columns.

	\[\begin{tikzcd}
		{\Hom(C,Z)} & {\Hom(C,X)} & {\Hom(C,Y)} & {\Ext^1(C,Z)} \\
		{\Hom(B,Z)} & {\Hom(B,X)} & {\Hom(B,Y)} & {\Ext^1(C,X)} \\
		{\Hom(A,Z)} & {\Hom(A,X)} & {\Hom(A,Y)} & {\Ext^1(A,Z)} \\
		{\Ext^1(C,Z)} & {\Ext^1(C,X)} & {\Ext^1(C,Y)} & \cdots
		\arrow[from=1-1, to=1-2]
		\arrow[from=1-1, to=2-1]
		\arrow["{g_*}", from=1-2, to=1-3]
		\arrow[from=1-2, to=2-2]
		\arrow["\delta", from=1-3, to=1-4]
		\arrow[from=1-3, to=2-3]
		\arrow[from=1-4, to=2-4]
		\arrow[from=2-1, to=2-2]
		\arrow["{f^*}", from=2-1, to=3-1]
		\arrow["{g_*}", from=2-2, to=2-3]
		\arrow["{f^*}", from=2-2, to=3-2]
		\arrow["\delta", from=2-3, to=2-4]
		\arrow["{f^*}", from=2-3, to=3-3]
		\arrow["{f^*}", from=2-4, to=3-4]
		\arrow[from=3-1, to=3-2]
		\arrow["\delta", from=3-1, to=4-1]
		\arrow["{g_*}", from=3-2, to=3-3]
		\arrow["\delta", from=3-2, to=4-2]
		\arrow["\delta", from=3-3, to=3-4]
		\arrow["\delta", from=3-3, to=4-3]
		\arrow[from=3-4, to=4-4]
		\arrow[from=4-1, to=4-2]
		\arrow["{g_*}", from=4-2, to=4-3]
		\arrow[from=4-3, to=4-4]
	\end{tikzcd}\]
	Our current objective is finding $\gamma\in \Hom(B,X)$ such that $f^* \gamma = \alpha$ and $g_* \gamma = \beta$, we will do so through some diagram chasing. We begin by noticing that \[g_*\delta \alpha = \delta g_* \alpha = \delta f^* \beta = 0. \] By lemma \ref{lem:ext-model} we know that ${\Ext^1(C,Z)} = 0$, therefore $g_* \colon{\Ext^1(C,X)} \longrightarrow {\Ext^1(C,Y)}$ is injective, thus $\delta \alpha = 0$ and by the exactness of the column we know that there exists $\gamma'\in \Hom(B,X)$ such that $f^* \gamma' = \alpha$. Given that $f^*(\beta - g_*\gamma')= f^*\beta -g_* \alpha = 0$, the morphism $\beta - g_*\gamma'$ must have a preimage $F$ in ${\Hom(C,Y)}$; using again lemma \ref{lem:ext-model}, we know that $g_*\colon {\Hom(C,X)} \longrightarrow  {\Hom(C,Y)}$ is surjective, and as a result there must be a morphism $G\in {\Hom(C,X)}$ such that $g_* G = F$. If we denote as $G'$ the image of $G$ in $\Hom(B,X)$, it is easy to check that $\gamma = \gamma' - G'$ is the morphism we are looking for.

\end{proof}



The objective of this section will be to prove \virgolette{Hovey’s Correspondence theeorem} (see section 2 of \cite{Hovey2002}), that shows a correspondence between abelian model structures and certain couples of cotorsion pairs. We will fix an abelian category $\C$ 

\begin{definition}[Thick class]
	A class of objects $W$ of $\C$ is said to be \textbf{thick} if it closed under retracts and whenever two out of three objects in a SES are in $W$ the third also is. 
\end{definition}

\begin{definition}[Hovey triple]
	Let $(C,W,F)$ be a triple of classes of objects in $\C$. We say that $(C,W,F)$ is an \textbf{Hovey triple} if $W$ is thick and $(C\cap W, F)$ and $(C, F\cap W)$ are complete cotorsion pairs.
\end{definition}



\begin{theorem}[Hovey's Correspondence - Part 1]
	If $\C$ is an abelian model category, $C$ its cofibrant objects, $F$ its fibrant objects and $W$ its acyclic objects, then $(C,W,F)$ is an Hovey triple.
\end{theorem}

\begin{proof}
	We begin by showing that $(C, F\cap W)$ is a complete cotorsion pair, given that proving it for $(C\cap W, F)$ is analogous. From lemma \ref{lem:ext-model}, it is immediate that $F\cap W \subseteq \rperp{C}$ and $C \subseteq \lperp{(F\cap W)}$, therefore to conclude the proof that $(C, F\cap W)$ is a cotorsion pair, we will also need to show that $\rperp{C} \subseteq  F\cap W$ and $\lperp{(F\cap W)} \subseteq C$; we will only prove the first of these two results (the other is dual). Let us now consider an object $B$ such that  $\Ext^1(A,B) = 0$ for every $A\in C$; then if we take any cofibration $f\colon X \longrightarrow Y$ we get an exact sequence
	\[\begin{tikzcd}
		0 & X & Y & {\coker(f)} & {0,}
		\arrow[from=1-1, to=1-2]
		\arrow["f", from=1-2, to=1-3]
		\arrow[from=1-3, to=1-4]
		\arrow[from=1-4, to=1-5]
	\end{tikzcd}\]
	where $\coker(f)$ must be a cofibrant object. Using the long exact sequence associated to the functor $\Hom({-},B)$ and the SES above, we get
	\[\begin{tikzcd}
		\cdots & {\Hom(X,B)} & {\Hom(Y,B)} & {\Ext^1(\coker(f),B)} & {\cdots;}
		\arrow[from=1-1, to=1-2]
		\arrow["{f^*}", from=1-2, to=1-3]
		\arrow[from=1-3, to=1-4]
		\arrow[from=1-4, to=1-5]
	\end{tikzcd}\]
	given that $\Ext^1(\coker(f),B)$ must be $0$, $f^*$ is surjective and therefore, by lemma \ref{lem:fib-to-surj}, $B$ must be in $F\cap W$. The last step we need is showing that this cotorsion pair is actually complete, but this is simply a consequence of (\textbf{MC4}): to show that $(C, F\cap W)$ has enough projectives, we can factor any initial map $0\longrightarrow X$ using an acyclic fibration $B \longrightarrow X$, where $B$ is cofibrant; taking the kernel of this map defines the exact sequence we need. Since the proof that shows that our cotorsion pair has enough injectives is identical, it must be complete.\\
	The last step that we need to complete this proof, is verifying that $W$ is a thick class. We know from axiom (\textbf{MC2}) that $W$ is closed under retracts, so we will only check the condition on SESes. 
	We begin by considering a SES as follows,
	\[\begin{tikzcd}
		0 & A & B & C & 0
		\arrow[from=1-1, to=1-2]
		\arrow["f", from=1-2, to=1-3]
		\arrow["g", from=1-3, to=1-4]
		\arrow[from=1-4, to=1-5]
	\end{tikzcd}\]
	then
	then we set


	\[\begin{tikzcd}
		0 & A & B & C & 0 \\
		0 & {A'} & {B'} & C & 0
		\arrow[from=1-1, to=1-2]
		\arrow["f", from=1-2, to=1-3]
		\arrow["i", from=1-2, to=2-2]
		\arrow["g", from=1-3, to=1-4]
		\arrow["j", from=1-3, to=2-3]
		\arrow[from=1-4, to=1-5]
		\arrow["{\id_C}", from=1-4, to=2-4]
		\arrow[from=2-1, to=2-2]
		\arrow["{f'}", from=2-2, to=2-3]
		\arrow["{g'}", from=2-3, to=2-4]
		\arrow[from=2-4, to=2-5]
	\end{tikzcd}\]
\end{proof}








\begin{theorem}[Hovey's Correspondence - Part 2]
	If $\C$ is an abelian model category and $(C,W,F)$ is an Hovey triple, then there exists an abelian model structure on $\C$ where $C$ are the cofibrant objects, $F$ are the fibrant objects and $W$ are the acyclic objects.
\end{theorem}








\begin{definition}[Small cotorsion pair]
	
\end{definition}





\begin{definition}
	
\end{definition}