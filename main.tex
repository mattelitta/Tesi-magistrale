\documentclass{article}
\usepackage[utf8]{inputenc}
\title{Corso STEM PNRR di Preparazione alle Competizioni di Matematica}
\date{}

\begin{document}

\maketitle

\section*{Obiettivo Generale}
Il progetto mira a potenziare le competenze matematiche degli studenti e prepararli per le competizioni matematiche nazionali e internazionali. Il corso è orientato a promuovere l'interesse e la passione per la matematica, sviluppando abilità di problem-solving avanzate.

\section*{Descrizione del Corso}
Il corso sarà strutturato in moduli tematici che copriranno argomenti fondamentali e avanzati di matematica, tra cui algebra, geometria, teoria dei numeri, combinatoria e analisi. Ogni modulo includerà lezioni teoriche, esercitazioni pratiche e simulazioni di gare matematiche.

\section*{Durata}
Il corso si svolgerà durante l'intero anno scolastico, con incontri settimanali di 2 ore ciascuno.

\section*{Metodologia}
\begin{itemize}
  \item \textbf{Lezioni interattive:} Utilizzo di metodologie didattiche innovative e interattive per rendere le lezioni coinvolgenti.
  \item \textbf{Esercitazioni pratiche:} Esercizi mirati e problemi di gara per sviluppare abilità di problem-solving.
  \item \textbf{Simulazioni di gare:} Simulazioni di competizioni matematiche per preparare gli studenti alle gare reali.
  \item \textbf{Tutoraggio:} Supporto individuale per gli studenti con maggiori difficoltà e per quelli più avanzati.
\end{itemize}

\section*{Valutazione}
Gli studenti saranno valutati attraverso prove scritte, esercizi pratici e simulazioni di gare matematiche.

\section*{Risultati Attesi}
\begin{itemize}
  \item Incremento delle competenze matematiche e delle abilità di problem-solving degli studenti.
  \item Maggiore interesse e passione per la matematica.
  \item Preparazione adeguata per competizioni matematiche nazionali e internazionali.
\end{itemize}


\section*{Conclusione}
Il corso STEM di preparazione alle competizioni di matematica rappresenta un'opportunità significativa per gli studenti di sviluppare competenze avanzate e di avvicinarsi al mondo delle competizioni matematiche, favorendo lo sviluppo di abilità fondamentali per il loro futuro accademico e professionale.

\end{document}